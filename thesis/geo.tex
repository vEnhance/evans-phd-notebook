\section{The geometric side}
\label{sec:geo}

\subsection{Rapoport-Zink spaces}
We briefly recall the theory of Rapoport-Zink spaces.
This follows the exposition in \cite[\S4.1]{ref:survey}.

Let $\breve F$ denote the completion of a maximal unramified extension of $F$,
and let $\FF$ denote the residue field of $\OO_{\breve F}$.
Suppose $S$ is a $\Spf \OO_{\breve F}$-scheme.
Then we can consider triples $(X, \iota, \lambda)$ consisting of the following data.
\begin{itemize}
  \ii $X$ is a formal $\varpi$-divisible $n$-dimensional $\OO_F$-module over $S$
  whose relative height is $2n$.

  \ii $\iota \colon \OO_E \to \End(X)$ is an action of $\OO_E$
  such that the induced action of $\OO_F$ on $\Lie X$
  is via the structure morphism $\OO_F \to \SO_S$.

  We require that $\iota$ satisfies the Kottwitz condition of signature $(n-1,1)$,
  meaning that for all $a \in \OO_E$,
  the characteristic polynomial of $\iota(a)$ on $\Lie X$
  is exactly \[ (T-a)^{n-1} (T-\bar a) \in \SO_S[T]. \]

  \ii $\lambda \colon X \to X^\vee$ is a principal $\OO_F$-relative polarization.

  We require that the Rosati involution of $\lambda$
  induces the map $a \mapsto \bar a$ on $\OO_F$
  (i.e.\ the nontrivial automorphism of $\Gal(E/F)$).
\end{itemize}
The triple is called supersingular if $X$ is a supersingular strict $\OO_F$-module.

For each $n \ge 1$, over $\FF$
we choose a supersingular triple $(\XX_n, \iota_{\XX_n}, \lambda_{\XX_n})$;
it's unique up to $\OO_E$-linear quasi-isogeny compatible with the polarization,
and refer to it as the \emph{framing object}.
We also let \[ \EE \coloneqq \XX_1. \]

We can now define the Rapoport-Zink space:
\begin{definition}
  For each $n \ge 1$, we let $\RZ_n$ denote the
  functor over $\Spf \OO_{\breve F}$ defined as follows.
  Let $S$ be an $\Spf \OO_{\breve F}$ scheme, and let
  $\ol S \coloneqq S \times_{\Spf \OO_{\breve F}} \Spec \FF$
  For every $\Spf \OO_{\breve F}$ scheme, we let $\RZ_n(S)$
  be the set of isomorphism classes of quadruples
  \[ (X, \iota, \lambda, \rho) \]
  where $(X, \iota, \lambda)$ is one of the triples as we described, and
  \[ \rho \colon X \times_S \ol S \to \XX_n \times_{\Spec \FF} \ol S \]
  is a \emph{framing}, meaning it is an height zero $\OO_F$-linear quasi-isogeny
  and satisfies
  \[ \rho^\ast((\lambda_{\XX_n})_{\ol S}) = \lambda_{\ol S}. \]
\end{definition}
Then $\RZ_n$ is formally smooth over $\OO_{\breve F}$ of relative dimension $n-1$.

Henceforth, we also make the following abbreviation.
\begin{definition}
  For integers $m$ and $n$
  \[ \RZ_{m,n} \coloneq \RZ_{m} \times_{\Spf \OO_{\breve F}} \RZ_n. \]
\end{definition}

\subsection{A realization of the non-split Hermitian space $\VV_n$ of dimension $n$}
At the same time, we can define the following Hermitian space.
\begin{definition}
  For each $n \ge 1$, let
  \[ \VV_n \coloneqq \Hom_{\OO_E}^\circ (\EE, \XX_n) \]
  which we call the space of special homomorphisms.
  When endowed with the form
  \[ \left< x,y \right> = \lambda_{\EE}^{-1} \circ y^\vee \circ \lambda_{\XX_n} \circ x
    \in \End_{E}^\circ(\EE) \simeq E \]
  it becomes an $n$-dimensional $E/F$-Hermitian space.
\end{definition}
\begin{proposition}
  Up to isomorphism, $\VV_n$ is the unique $n$-dimensional
  nondegenerate non-split $E/F$-Hermitian space.
\end{proposition}
\begin{proof}
  \todo{ref}
\end{proof}

\subsection{Euler-Poincar\'{e} characteristic}
In what follows, $\chi$ denotes the Euler-Poincar\'{e} characteristic,
meaning that if $X$ is a formal scheme over $\Spf \OO_{\breve F}$
then given a finite complex $\mathcal{F}$ of $\SO_X$-modules we set
\[ \chi_X(\mathcal{F}) = \sum_i \sum_j (-1)^{i+j}
  \operatorname*{len}_{\OO_{\breve F}} H^j(X, H_i(\mathcal F)) \]
provided all the lengths are finite.

\subsection{Hecke operator}
We need to define $\TT_{f'} \coloneqq \TT_{\mathbf{1}_{K'^\flat} \otimes f'}$.
\todo{help}

\subsection{Intersection number for the group version of AFL}
\begin{definition}
  Following \cite[\S6]{ref:AFLspherical}, we define the intersection number as
  \[
    \Int((1,g), \mathbf{1}_{K'^\flat} \otimes f)
    \coloneqq \chi_{\RZ_{n-1,n}} \left( \SO_{\TT_f (\Delta_{\RZ_{n-1}})}
      \mathop{\otimes}^{\LL}_{\SO_{\RZ_{n-1,n}}} \SO_{(1,g) \cdot \Delta_{\RZ_{n-1}}} \right).
  \]
\end{definition}

\subsection{Intersection number for the semi-Lie version of AFL}
In what follows, let
\[ \Gamma_g \subseteq \RZ_{n,n} \]
denote the graph of the automorphism of $\RZ_n$ induced by $g$.
\begin{definition}
  \[
    \Int((g,u), f)
    \coloneqq \chi_{\RZ_{n,n}} \left( \SO_{\Gamma_g}
      \mathop{\otimes}^{\LL}_{\SO_{\RZ_{n,n}}} \SO_{\Delta_{\ZD(u)}}
      \mathop{\otimes}^{\LL}_{\SO_{\RZ_{n,n}}} \SO_{\TT_f \Delta_{\RZ_n}} \right).
  \]
\end{definition}
\todo{you would think this is bilinear}
