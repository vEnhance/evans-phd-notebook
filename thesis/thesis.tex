% !TEX encoding = UTF-8 Unicod

% Based on MIT-Thesis.tex, a LaTeX template for formatting an MIT thesis with the mitthesis class.
% Version: 1.15, 2024/08/31
% Author: John H. Lienhard, copyright 2024. Reuse under the MIT license: https://ctan.org/license/mit
% Documentation is here: https://ctan.org/pkg/mitthesis
\DocumentMetadata{
  pdfstandard = a-2b,
  pdfversion  = 1.7,
  lang    = en-US,
}

\documentclass[twoside]{mitthesis}
% Preamble - Evan's thesis version

%%% Load packages
\usepackage{hyperref}
\usepackage[obeyFinal,textsize=scriptsize,shadow]{todonotes}
\usepackage[shortlabels]{enumitem}
\usepackage[usenames,dvipsnames,svgnames]{xcolor}
\usepackage{amsaddr}
\usepackage{amssymb}
\usepackage{booktabs}
\usepackage{cleveref}
\usepackage{derivative}
\usepackage{graphicx}
\usepackage{mathdots}
\usepackage{mathrsfs}
\usepackage{mathtools}
\usepackage{microtype}

\usepackage{asymptote}
\usepackage{tikz-cd}
\usetikzlibrary{decorations.pathmorphing}

\allowdisplaybreaks

\usepackage[backend=biber,backref=true,style=alphabetic]{biblatex}

\newtheorem{theorem}{Theorem}[section]
\newtheorem{lemma}[theorem]{Lemma}
\newtheorem{proposition}[theorem]{Proposition}
\newtheorem{corollary}[theorem]{Corollary}

\theoremstyle{definition}
\newtheorem{assume}[theorem]{Assumption}
\newtheorem{definition}[theorem]{Definition}
\newtheorem{example}[theorem]{Example}
\newtheorem{ques}[theorem]{Question}
\newtheorem{claim}[theorem]{Claim}
\newtheorem{conjecture}[theorem]{Conjecture}
\newtheorem{remark}[theorem]{Remark}
\newtheorem{question}[theorem]{Question}

%%% Macros
\providecommand{\ol}{\overline}
\providecommand{\ul}{\underline}
\providecommand{\wt}{\widetilde}
\providecommand{\wh}{\widehat}
\providecommand{\eps}{\varepsilon}
\providecommand{\half}{\frac{1}{2}}
\providecommand{\inv}{^{-1}}
\newcommand{\dang}{\measuredangle} %% Directed angle
\providecommand{\CC}{\mathbb C}
\providecommand{\FF}{\mathbb F}
\providecommand{\NN}{\mathbb N}
\providecommand{\QQ}{\mathbb Q}
\providecommand{\RR}{\mathbb R}
\providecommand{\ZZ}{\mathbb Z}
\providecommand{\ts}{\textsuperscript}
\providecommand{\dg}{^\circ}
\providecommand{\ii}{\item}
\newcommand{\surjto}{\twoheadrightarrow}

\DeclareMathOperator*{\Arch}{ARCH}
\DeclareMathOperator{\BC}{BC}
\DeclareMathOperator{\GL}{GL}
\DeclareMathOperator{\Hom}{Hom}
\DeclareMathOperator{\Int}{Int}
\DeclareMathOperator{\Mat}{Mat}
\DeclareMathOperator{\Norm}{N}
\DeclareMathOperator{\Orb}{Orb}
\DeclareMathOperator{\SL}{SL}
\DeclareMathOperator{\Sat}{Sat}
\DeclareMathOperator{\Sym}{Sym}
\DeclareMathOperator{\U}{U}
\DeclareMathOperator{\Vol}{Vol}
\DeclareMathOperator{\antidiag}{antidiag}
\DeclareMathOperator{\diag}{diag}
\DeclareMathOperator{\id}{id}
\DeclareMathOperator{\rproj}{proj}

\newcommand{\cc}{\mathbf c}
\newcommand{\nn}{\mathbf n}
\newcommand{\uu}{\mathbf u}
\newcommand{\vv}{\mathbf v}
\newcommand{\OO}{\mathcal O}
\newcommand{\HH}{\mathcal H}
\newcommand{\VV}{\mathbb V}

\newcommand{\BG}{\mathbf{G}}
\newcommand{\BK}{\mathbf{K}}
\newcommand{\BM}{\mathbf{M}}
\newcommand{\BN}{\mathbf{N}}
\newcommand{\BP}{\mathbf{P}}
\newcommand{\BS}{\mathbf{S}}

\newcommand{\oneV}{\mathbf{1}_{\OO_F^n \times (\OO_F^n)^\vee}}

\hypersetup{%
  pdfsubject={Explicit formulas for weighted orbital integrals for AFL},
  % Change this to briefly state topic of your thesis

  pdfkeywords={Evan Chen,arithmetic fundamental lemma,orbital integral},
  % Add keywords that will help search engines and libraries to find your work.
  % Includes the name[s] of the author[s]
  % (If you used \DocumentMetadata at line 14, you can just put "\CopyrightAuthor," for the names.)

  pdfurl={},
  % If you have a url for the thesis, put it here. Otherwise delete this.
  % (MIT Libraries will put your thesis in DSPACE with a persistent url after you submit it.)

  pdfcontactemail={evanchen@alum.mit.edu},
  % You can put a [permanent] email address into the metadata, if you like. Otherwise delete this.
  pdfauthortitle={},
  % If you have a title, you can include it here.
}

\addbibresource{refs.bib}
\usepackage{setspace}
\usepackage{draftwatermark}

\SetWatermarkAngle{32}
\SetWatermarkLightness{0.90}
\SetWatermarkFontSize{32pt}
\SetWatermarkScale{4}
\SetWatermarkText{DRAFT}

\begin{document}

\title{(DRAFT) Explicit formulas for weighted orbital integrals for the
  inhomogeneous and semi-Lie arithmetic fundamental lemmas
  conjectured for the full spherical Hecke algebra}

\Author{Evan Chen}{Department of Mathematics}
\Degree{Doctor of Philosophy in Mathematics}{Department of Mathematics}
\Supervisor{Wei Zhang}{Professor of Mathematics}
%\Acceptor{Davesh Maulik}{Professor of Mathematics}{Chairman, Department Committee on Graduate Studies}
\Acceptor{\bfseries STILL A DRAFT LMAO}{SO LIKE NOT ACCEPTED YET}{SERIOUSLY HOLD YOUR HORSES}
\DegreeDate{February}{2025}
\ThesisDate{January 10, 2025} % Date that final thesis is submitted to department

% Omit this command if you are not using a cc license.
\CClicense{CC BY-NC-ND 4.0}{https://creativecommons.org/licenses/by-nc-nd/4.0/}

\maketitle

\begin{abstract}
  As an analog to the Jacquet-Rallis fundamental lemma that appears in the
  relative trace formula approach to the Gan-Gross-Prasad conjectures,
  the arithmetic fundamental lemma was proposed by W.\ Zhang and used in an approach
  to the arithmetic Gan-Gross-Prasad conjectures.
  Both the Jacquet-Rallis fundamental lemma and arithmetic fundamental lemma
  were recently generalized to conjectural statements that hold
  for an arbitrary function in the associated spherical Hecke algebras.
  This paper produces explicit formulas for the weighted orbital integrals that appear
  in certain cases of the arithmetic fundamental lemma for the spherical Hecke algebra,
  thus verifying the conjecture in some particular cases.
\end{abstract}

\onehalfspacing % okay i decided double spacing just looks too stupid

%% acknowledgments.tex

% From mitthesis package
% Version: 1.02, 2024/06/19
% Documentation: https://ctan.org/pkg/mitthesis

\chapter*{Acknowledgments}
\pdfbookmark[0]{Acknowledgments}{acknowledgments}

I thank my advisor Wei Zhang for suggesting this project,
for his infinite patience and kindness throughout my entire time during graduate school,
and his seemingly instantaneous response times at all hours of the day
to the many, many stupid questions I asked.
Without the encouragement and support from Wei,
I would certainly never have completed my thesis or even passed my qualifying exams.

I also thank MIT itself for serving as my home for the last ten years,
and the math department for supporting me since I started graduate school.
I'd like to thank Barbara Peskin and Bjorn Poonen in particular
for some words of encouragement during my earlier years of PhD study,
and Theresa Cummings and Michele Gallarelli for their assistance throughout my studies.

I'd like to acknowledge my mentors during my undergraduate years,
particularly Joe Gallian and Ken Ono who had a great role in my journey.
Plus another thank-you to my mentors from high school, particularly
Cheryl Chiu, Zuming Feng, Beth Rothfuss, Zvezda Stankova, Paul Zeitz, and Yan Zhang;
Steve Dunbar's team in Lincoln, Nebraska which organized the USA Math Olympiad
and the Math Olympiad Summer Program;
and the coaches from the Taiwan IMO selection and training.

I thank the organizers and attendees of the fall 2024
learning seminar on arithmetic inner product formula
for inviting me to speak about my work-in-progress, and their helpful comments on it
leading directly to improvements to this thesis.

I thank Mark Sellke for proofreading a draft of this thesis
and finding several corrections.

Finally, I thank Ben Howard and Zhiwei Yun for serving on my thesis committee,
as well as David Vogan and Nike Sun for serving on my qualifying exams committee.

This work was partially supported by NSF GRFP under grant numbers 1745302 and 2141064.


\tableofcontents
\listoffigures
\listoftables

\section{Introduction}
Throughout this whole paper, $p > 2$ is a prime,
$F$ is a finite extension of $\QQ_p$,
and $E/F$ is an unramified quadratic field extension.

\subsection{History and motivation for the arithmetic fundamental lemma}
The primary motivation for this paper arises from
the study of conjectured variants of the arithmetic fundamental lemma
for spherical Hecke algebras proposed in \cite{ref:AFLspherical}.
This section briefly provides an overview of the historical context
that led to the formulation of these conjectures.
This history is also summarized in \Cref{fig:history}.
A more detailed account can be found in \cite{ref:survey}.

\begin{figure}[ht]
  \centering
  \begin{tikzcd}
      \text{\cite{ref:waldspurger}} \ar[d] \\
      \text{\cite{ref:GP1,ref:GP2}} \ar[d] \\
    \text{GGP \cite{ref:GGP}}
      \ar[d, dotted, leftrightarrow, "\text{analog}"]
      & \ar[l, Rightarrow, "\text{used to prove}"] \text{FL \cite{ref:JR}}
        \ar[d, dotted, leftrightarrow, "\text{analog}"] \ar[r]
      & \text{\cite{ref:leslie}}
        \ar[d, dotted, leftrightarrow, "\text{analog}"] \\
    \text{Arith.\ GGP \cite{ref:GGP}}
      & \ar[l, Rightarrow, "\text{used to prove}"] \text{AFL \cite{ref:AFL}} \ar[r]
      & \text{\cite{ref:AFLspherical}} \\
    \text{\cite{ref:GZshimura}} \ar[u] \\
    \text{\cite{ref:gross_zagier}} \ar[u]
  \end{tikzcd}
  \caption{The history behind the fundamental lemma and its arithmetic counterpart.
    Unlabeled arrows denote generalizations.}
  \label{fig:history}
\end{figure}

\subsubsection{The GGP conjectures, and the fundamental lemma of Jacquet-Rallis}
In modern arithmetic geometry, a common theme is that there are deep connections
between geometric data with the values of related $L$-functions.

This story begins with a result of
Waldspurger \cite{ref:waldspurger} which showed a formula
relating the nonvanishing of an automorphic period integral
to the central value of the same $L$-functions.
Later, a conjecture that generalizes Waldspurger's formula
was proposed by Gross-Prasad in \cite{ref:GP1,ref:GP2}.
This was further generalized to a series of conjectures
now known as the Gan-Gross-Prasad (GGP) conjectures,
which were proposed in 2012 in \cite{ref:GGP};
they generalize the Gross-Prasad conjecture to different classical groups.
Specifically, the GGP conjecture predict the nonvanishing of a period integral
based on the values of the $L$-function of a certain cuspidal automorphic representation.

In 2011, Jacquet-Rallis \cite{ref:JR} proposed an approach to the Gross-Prasad conjectures
for unitary groups via a relative trace formula (RTF).
The idea is to compare a RTF for the general linear group to one for a unitary group.
This approach relies on a so-called \emph{fundamental lemma},
which links values of certain orbital integrals
over two reductive groups over a non-Archimedean local field.

Let's be a bit more precise about what this fundamental lemma says.
Let $V_0$ denote a split $E/F$-Hermitian space of dimension $n$,
fix a unit vector $w_0$ in it,
and let $V_0^\flat$ be the orthogonal complement of the span of $w_0$.
Let $G'^\flat \coloneqq \GL_{n-1}(E)$, $G' \coloneqq \GL_n(E)$,
$G^\flat \coloneqq \U(V_0^\flat)(F)$ and $G \coloneqq \U(V_0)(F)$.
For certain
\[ \gamma \in G'^\flat \times G', \qquad g \in G^\flat \times G \]
the Jacquet-Rallis fundamental lemma proposes a relation between two orbital integrals.
Specifically, it supplies a relation between
\begin{itemize}
\item the orbital integral of $\gamma$ with respect to
  the indicator function $\mathbf{1}_{K'^\flat \times K'}$
  of the natural hyperspecial compact subgroup
  \[ K'^\flat \times K' \subset G'^\flat \times G' = \GL_{n-1}(E) \times \GL_n(E); \]
  and
\item the orbital integral of $g$ with respect to
  the indicator function $\mathbf{1}_{K^\flat \times K}$
  of the natural hyperspecial compact subgroup
  \[ K^\flat \times K \subset G^\flat \times G = \U(V_0^\flat)(F) \times \U(V_0)(F). \]
\end{itemize}
In other words, it states that
\begin{equation}
  \Orb(\gamma, \mathbf{1}_{K'^\flat \times K'}) = \omega(\gamma) \Orb(g, \mathbf{1}_{K^\flat \times K})
  \label{eq:old_FL}
\end{equation}
where $\omega(\gamma)$ is a suitable \emph{transfer factor}.
The fundamental lemma has since been proved completely;
a local proof was given by Beuzart-Plessis \cite{ref:BeuzartPlessis}
while a global proof was given for large characteristic by W.\ Zhang \cite{ref:Wei2021}.

\subsubsection{The arithmetic GGP conjectures, and the arithmetic fundamental lemma.}
At around the same time Waldspurger's formula was published,
Gross-Zagier \cite{ref:gross_zagier} proved a formula
relating the height of Heegner points
on certain modular curves to the derivative at $s=1$ of certain $L$-functions.
The Gross-Zagier formula was then generalized over several decades,
culminating in \cite{ref:GZshimura} where the formula is established
for Shimura curves over arbitrary totally real fields.

An arithmetic analogue of the original Gan-Gross-Prasad conjectures,
which we henceforth refer to as \emph{arithmetic GGP} \cite{ref:GGP},
can then be formulated, generalizing Gross-Zagier's formula.
Here the modular curves in Gross-Zagier
are replaced with higher dimensional Shimura varieties.
Rather than the period integrals considered previously,
one instead takes intersection numbers of cycles on some Shimura varieties.
Specifically, if one considers the Shimura variety associated to a classical group,
the arithmetic GGP conjecture predicts a relation between intersection numbers
on this Simura variety with the central derivative of automorphic $L$-functions.

By analogy to the work Jacquet-Rallis \cite{ref:JR},
the arithmetic GGP conjectures should have a corresponding
\emph{arithmetic fundamental lemma} (henceforth AFL),
which was proposed by W.\ Zhang \cite{ref:AFL}.
The arithmetic fundamental lemma then relates the derivative
of the orbital integral with respect to the indicator function
$\mathbf{1}_{K'^\flat \times K'} \in \HH(G'^\flat \times G, K'^\flat \times K')$, that is
\[ \left. \pdv{}{s} \right\rvert_{s=0} \Orb(\gamma, \mathbf{1}_{K'^\flat \times K'}) \]
for $\gamma \in G'^\flat \times G'$,
to arithmetic intersection numbers on a certain Rapoport-Zink formal moduli space.
The AFL in \cite{ref:AFL} has since been proven over $p$-adic fields for any prime $p$ in
Mihatsch-Zhang \cite{ref:MZ2021}, W.\ Zhang \cite{ref:Wei2021}, Z.\ Zhang \cite{ref:Zhiyu}.

\subsubsection{Generalizations of FL and AFL to spherical Hecke algebras}
Recently it was shown by Leslie \cite{ref:leslie} that in fact
\eqref{eq:old_FL} holds in greater generality where the indicator function
$\mathbf{1}_{K^\flat \times K}$ can be replaced by any element
in the spherical Hecke algebra $\varphi \in \HH(G'^\flat \times G', K'^\flat \times K')$.
In that case, $\mathbf{1}_{K'^\flat \times K}$ needs to be replaced
by the corresponding element $\varphi'$ under a certain base change homomorphism
\[ \varphi \mapsto \varphi' \qquad
  \HH(G'^\flat \times G', K'^\flat \times K') \to \HH(G^\flat \times G, K^\flat \times K). \]
In that case, the identity \eqref{eq:old_FL} still hold as
\begin{equation}
  \Orb(g, \varphi) = \omega(\gamma) \Orb(\gamma, \varphi').
  \label{eq:eq:leslie_FL}
\end{equation}
To complete the analogy illustrated in \Cref{fig:history},
there should thus be a generalization of the AFL in which
$\mathbf{1}_{K'^\flat \times K}$ is replaced by any element of the Hecke algebra
$\HH(G'^\flat \times G, K'^\flat \times K)$.
This formula is proposed by \cite{ref:AFLspherical},
and is the primary focus of this paper; we discuss it in the next section.

\subsection{The inhomogeneous version of the arithmetic fundamental lemma for spherical Hecke algebras}
Retain the notation $G' \coloneqq \GL_n(E)$, and $G \coloneqq \U(V_0)(F)$,
with $K' \subset G'$ and $K \subset G$ the natural hyperspecial compact subgroups.
For concreteness, we focus on the inhomogeneous version
of the arithmetic fundamental lemma, which is \cite[Conjecture 6.2.1]{ref:AFLspherical},
which allows us to deal with just $G'$ instead of $G'^\flat \times G'$, etc.

Let $q$ denote the residue characteristic of $F$.
Also define the symmetric space
\[ S_n(F) \coloneqq \left\{ g \in \GL_n(E) \mid g \bar{g} = \id_n \right\}. \]
In this case, the orbital integrals in the inhomogeneous AFL
are stated more naturally in terms of $f'$ contained in a certain
$\HH(G',K')$-module which we will denote $\HH(S_n(F), K')$;
it is defined later in \Cref{sec:background}.

\begin{conjecture}
  [{\cite[Conjecture 6.2.1]{ref:AFLspherical}}]
  \label{conj:inhomog}
  Let $f \in \HH(G, K)$ be any element of the Hecke algebra,
  and let $f' \in \HH(S_n(F), K')$ be its image
  under a base change $\BC_{S_n}^{\eta^{n-1}}$ defined in \Cref{sec:satake}.
  Then for certain pairs $g \in G$\todo{in $G$, right? check with Wei}
  and $\gamma \in S_n(F)$, we have
  \begin{equation}
    \Int\left( (1,g), \mathbf{1}_{K'^\flat \otimes f} \right) \log q
    = -\omega(\gamma) \left. \pdv{}{s} \right\rvert_{s=0} \Orb(\gamma, f').
    \label{eq:inhomog}
  \end{equation}
\end{conjecture}

At present, the (inhomogeneous) AFL is the case where $f = \mathbf{1}_K$ and is thus proven.
The generalized conjecture is also proved in full for
$n = 2$ in \cite[Theorem 1.0.1]{ref:AFLspherical}
(in that reference, our $n$ denotes their $n+1$).
The part of the calculation involving orbital integral has two parts:
\begin{itemize}
  \ii The calculation makes $\BC_{S_{n}}^{\eta^{n-1}}$
  completely explicit in a natural basis for $n = 2$.
  The result is \cite[Lemma 7.1.1]{ref:AFLspherical}.

  \ii The calculation makes explicit the value of the orbital integral
  \[ \Orb(\gamma, f') \]
  for any $\gamma \in S_n(F)$ and $f' \in \HH(S_n(F), K')$,
  in terms of invariants of $\gamma$ and a decomposition of $f'$ in a natural basis.
  The result is \cite[Proposition 7.3.2]{ref:AFLspherical}.
\end{itemize}
Combining these two (hence obtaining the right-hand side of \eqref{eq:inhomog})
with a calculation of intersection numbers in \cite[Corollary 7.4.3]{ref:AFLspherical}
(which is the left-hand side of \eqref{eq:inhomog})
shows that \Cref{conj:inhomog} holds for $n = 2$,
cf.\ \cite[Theorem 7.5.1]{ref:AFLspherical}.

\subsection{Results}
The goal of this paper is to perform the analogous calculations
that correspond to \cite[Lemma 7.1.1 and Proposition 7.3.2]{ref:AFLspherical}
in two new situations:
\begin{itemize}
  \ii The case $n = 3$ as in \Cref{conj:inhomog},
  which involves the orbits of $\GL_2(F)$ on a certain symmetric space $S_3(F)$.

  \ii A parallel Fourier-Jacobi version where one instead takes the action
  of $\GL_2(F)$ on $S_2 \times F^2 \times F_2$.
\end{itemize}
The methods, which are local in nature,
are rather similar to those employed in \cite{ref:AFL},
which shows how to compute the orbital integral for $n = 3$ when $f = \mathbf{1}_{K'}$.
We state these results as follows.

\begin{theorem}
\end{theorem}
\todo{write summary theorem}

Besides providing evidence for \Cref{conj:inhomog},
these orbital integral formulas also establish some other results.
For example, the formula above
proves the $n = 3$ case of \cite[Conjecture 8.2.1]{ref:AFLspherical}
and an analogous version; we state this below.

\begin{theorem}
\end{theorem}
\todo{need to figure out why there are all these semisimple restrictions}

\subsection{Roadmap}
The rest of the paper is organized as follows.
We provide some necessary background and notation in \Cref{sec:background}.
Then, in \Cref{sec:orbital0} we describe our results on the orbital integral,
and dedicate \Cref{sec:orbital1,sec:orbital2} to their proof.
And in \Cref{sec:satake} we provide the explicit base change formula.

\subsection{Acknowledgments}
\todo{write this}


\chapter{General background}
\label{ch:background}

\section{Notation}
We provide a glossary of notation that will be used in this paper.
As mentioned in the introduction, $p > 2$ is a prime,
$F$ is a finite extension of $\QQ_p$,
and $E/F$ is an unramified quadratic field extension.

\begin{itemize}
  \ii For any $a \in E$, we let $\bar a$ denote the image of $a$
  under the nontrivial automorphism of $\Gal(E/F)$.
  (Hence $a = \bar a$ exactly when $a \in F$.)
  \ii Fix $\eps \in \OO_F^\times$ such that $E = F[\sqrt{\eps}]$.
  \ii Denote by $\varpi$ a uniformizer of $\OO_F$, such that $\bar \varpi = \varpi$.
  \ii Let $q \coloneqq |\OO_F/\varpi|$ be the residue characteristic.
  (Hence $|\OO_E / \varpi| = q^2$.)
  \ii Let $v$ be the associated valuation for $\varpi$.
  \ii Let $\eta$ be the quadratic character attached to $E/F$ by class field theory,
  so that $\eta(x) = -1^{v(x)}$.
  \ii Set $G' \coloneqq \GL_n(E)$.
  \ii Set $K' \coloneqq \GL_n(\OO_E) \subseteq G'$ as the hyperspecial maximal compact subgroup of $G$.
  \ii $V_0$ denotes a split $E/F$-Hermitian space of dimension $n$ (unique up to isomorphism).
  \ii Let $\beta$ denote the $n \times n$ antidiagonal matrix
  \[ \beta \coloneqq \begin{bmatrix} && 1 \\ & \iddots \\ 1 \end{bmatrix} \]
  and pick the basis of $V_0$ such that the Hermitian form on $V_0$ is given by
  \[ V_0 \times V_0 \to E \qquad (x,y) \mapsto x^\ast \beta y. \]
  \ii Set
  \[ G \coloneqq \U(V_0) = \{ g \in \GL_n(\OO_E) \mid g^\ast \beta g = \beta\} \]
  the unitary group over $V_0$.
  Note that $\beta$ is \emph{antidiagonal}, in contrast to the convention $\beta = \id_n$
  that is often used for unitary matrices with entries in $\CC$.
  \ii Set
  \[ K \coloneqq G \cap \GL_n(\OO_E) \]
  as the natural hyperspecial maximal compact subgroup.
  \ii Let $\VV_n$ denote the non-split $E/F$-Hermitian space of dimension $n$
  (unique up to isomorphism), and $\U(\VV_n)$ the corresponding unitary group.
  This space will be realized in \Cref{ch:geo}.
\end{itemize}

\section{The spaces $S_n(F)$ and $S_n(F) \times V'_n$}
For the analytic side of the two AFL conjectures we investigate,
the following two spaces will be used as inputs to our orbital integrals.
\begin{definition}
  [{\cite[(4.10)]{ref:highdim2024}}]
  We define the symmetric space
  \[ S_n(F) \coloneqq \left\{ g \in \GL_n(E) \mid g \bar g = \id_n \right\}. \]
  It has a natural left action of $\GL_n(E)$ by
  \begin{align*}
    \GL_n(E) \times S_n(F) &\to S_n(F) \\
    g \cdot \gamma &\coloneqq g \gamma \bar g^{-1}.
  \end{align*}
\end{definition}

\begin{definition}
  [{\cite[(4.11)]{ref:highdim2024}}]
  We set
  \[ V'_n(F) \coloneqq F^n \times (F^n)^\vee \]
  where $-^\vee$ denotes the $F$-dual space, i.e., $(F^n)^\vee = \Hom_F(F^n, F)$.
  Then we may also consider the augmented space
  \[ S_n(F) \times V'_n(F) \]
  If we identify $F^n$ with column vectors of length $n-1$ and $(F^n)^\vee$
  with row vectors of length $n$ then we have a left action of $\GL_n(F)$ by
  \begin{align*}
    \GL_n(F) \times (S_n(F) \times V'_n)(F)
    &\to S_n(F) \times V'_n(F) \\
    h \cdot (\gamma, \uu, \vv^\top)
    &\coloneqq (h \gamma h^{-1}, h\uu, \vv^\top h^{-1}).
  \end{align*}
  Note that according to the embedding
  \begin{align*}
    S_n(F) \times V'_n(F)
    &\hookrightarrow \GL_{n+1}(E) \\
    (\gamma, \uu, \vv^\top)
    &\mapsto \begin{bmatrix} \gamma & \uu \\ \vv^\top & 0 \end{bmatrix}
  \end{align*}
  we can consider elements of $S_n(F) \times V'_n(F)$ as elements of $\GL_{n+1}(E)$ too.
  In that case the action of $h \in \GL_{n+1}(F)$
  coincides with $h \cdot g \mapsto hg\bar{h}^{-1}$ as well.
\end{definition}

\begin{definition}
  For brevity, let
   \[ K'_S \coloneqq S_n(F) \cap \GL_n(\OO_F). \]
\end{definition}

\section{Definition of Hecke algebra}
We reminder the reader the definition of a Hecke algebra.
For this subsection, $G$ will denote \emph{any}
unimodular locally compact topological group,
and $K$ any closed subgroup of $G$.

\begin{definition}
  The \emph{Hecke algebra}
  \[ \HH(G, K) \coloneqq \QQ[K \backslash G \slash K] \]
  is defined as the space of compactly supported $K$-binvariant
  locally constant functions on $G$.

  Given two such functions $f_1$ and $f_2$ in $\HH(G,K)$,
  one can consider define the convolution
  \[ (f_1 \ast f_2)(g) \coloneqq \int_G f_1(g\inv x) f_2(x) \; \odif x \]
  which makes $\HH(G, K)$ into a $\QQ$-algebra,
  whose identity element is $\mathbf{1}_K$.
\end{definition}
In the case where $G$ is a reductive Lie group and
$K$ is the maximal compact subgroup
(or more generally whenever $(G,K)$ is a Gelfand pair),
this Hecke algebra is actually commutative.

\section{The specific Hecke algebras $\HH(G',K')$ and $\HH(G,K)$
  for $G' = \GL_n(E)$ and $G = \U(V_0)$, and the module $\HH(S_n(F), K')$}
For our purposes, we define two Hecke algebras:
\begin{align*}
  \HH(G', K') &\coloneqq \HH(\GL_n(E), \GL_n(\OO_E)) \\
  \HH(G, K) &\coloneqq \HH(\U(V_0), \U(V_0) \cap \GL_n(\OO_E))
\end{align*}

Now the symmetric space $S_n(F)$ is not a group,
so it does not make sense to define the same thing here.
Nevertheless, we introduce
\[ \HH(S_n(F), K') \coloneqq \mathcal C_{\mathrm{c}}^{\infty}(S_n(F))^{K'} \]
as the set of smooth compactly supported functions on $S_n(F)$
which are invariant under the action of $K' \subseteq G'$;
this is an $\HH(G', K')$-module, where the action of $f \in \HH(G', K')$ is given by
\[ f \cdot \varphi = {?} \]
\todo{how do I set RHS}
This does \textbf{not} have a multiplication structure at the moment, \emph{a priori},
although later we will see how one could be imposed.

Throughout this paper, to be unambiguous with the notation, we denote
\begin{itemize}
  \ii elements of $\HH(G,K)$ using $f$ or $f_i$ or similar
    (i.e.\ lowercase Roman letters);
  \ii elements of $\HH(G',K')$ by $f'$ or $f'_i$ or similar
    (i.e.\ lowercase Roman letters with apostrophes);
  \ii elements of $\HH(S_n(F),K')$ by $\phi$ or $\phi'_i$
    (i.e.\ lowercase Greek letters).
\end{itemize}

%Similarly, consider $S_n(F) \times V_n$.
%We set
%\[ \HH(S_n(F) \times V_n', {?}) \coloneqq \mathcal C_{\mathrm{c}}^{\infty}(S_n(F) \times V_n')^{K'} \]
%as the set of smooth compactly supported functions on $S_n(F) \times V_n'$
%which are invariant under the action of ${?} \subseteq G'$.
%\todo{This should be $\GL_n(\OO_F)$? Does it need another name?}
%This is an \dots

\section{Arches}
We introduce one more piece of notation for a common shape that our answers will take.

\begin{definition}
  Suppose $\{a_0, a_0 + 1, \dots, a_1\}$ is an interval of integers for some $a_0 \le a_1$,
  and consider two more integers $w_1$ and $w_2$ such that $w_1 + w_2 \le \frac{a_1-a_0}{2}$.
  Then we can define a piecewise linear function
  \[ \Arch_{[a_0, a_1]}(w_1, w_2) \colon \{a_0, a_0+1, \dots, a_1\} \to \ZZ_{\ge 0} \]
  according to the following definition:
  \[
    k \mapsto
    \begin{cases}
      k - a_0 & \text{if }a_0 \le k \le a_0 + w_1 \\
      w_1 + \left\lfloor \frac{k-(a_0+w_1)}{2} \right\rfloor & \text{if } a_0 + w_1 \le k \le a_0 + w_1 + w_2 \\
      w_1 + \left\lfloor \frac{w_2}{2} \right\rfloor & \text{if } a_0 + w_1 + w_2 \le k \le a_1 - (w_1 + w_2)\\
      w_1 + \left\lfloor \frac{(a_1-w_1) - k}{2} \right\rfloor & \text{if } a_1 - (w_1 + w_2) \le k \le a_1 - w_1 \\
      a_1 - k & \text{if }a_1 - w_1 \le k \le a_1.
    \end{cases}
  \]
\end{definition}
The nomenclature is meant to be indicative of the shape of the graph,
which looks a little bit like an arch.
It is a function symmetric around $\frac{a_0+a_1}{2}$ defined piecewise.
The function grows linearly with slope $1$ at the far left for $w_1$ steps,
then changes to slope $1/2$ for $w_2$ steps (rounding down),
before stabilizing, then doing the symmetric descent on the right half.
\begin{figure}[ht]
  \begin{center}
  \begin{asy}
    size(12cm);
    draw((-1,0)--(20,0));
    draw((0,0)--(3,3)--(7,5)--(12,5)--(16,3)--(19,0), lightred);
    draw((3,3)--(3,0), grey);
    draw((7,5)--(7,0), grey);
    draw((12,5)--(12,0), grey);
    draw((16,3)--(16,0), grey);
    real eps = 0.3;
    void brack(string s, real x0, real x1) {
      draw((x0+0.1,-eps)--(x0+0.1,-2*eps)--(x1-0.1,-2*eps)--(x1-0.1,-eps), blue);
      label(s, ((x0+x1)/2, -2*eps), dir(-90), blue);
    }
    brack("$w_1 = 3$", 0, 3);
    brack("$w_2 = 4$", 3, 7);
    brack("$w_2 = 4$", 12, 16);
    brack("$w_1 = 3$", 16, 19);

    dotfactor *= 1.5;
    dot("$(0,0)$", (0,0), dir(225));
    dot((1,1), red);
    dot((2,2), red);
    dot("$(3,3)$", (3,3), dir(135));
    dot((4,3), red);
    dot((5,4), red);
    dot((6,4), red);
    dot("$(7,5)$", (7,5), dir(90));
    dot((8,5), red);
    dot((9,5), red);
    dot((10,5), red);
    dot((11,5), red);
    dot("$(12,5)$", (12,5), dir(90));
    dot((13,4), red);
    dot((14,4), red);
    dot((15,3), red);
    dot("$(16,3)$", (16,3), dir(45));
    dot((17,2), red);
    dot((18,1), red);
    dot("$(19,0)$", (19,0), dir(315));

    label(rotate(45)*"Slope $+1$", (1.5,1.5), dir(135), lightred);
    label(rotate(26.57)*"Slope $+\frac12$", (5,4), dir(125), lightred);
    label(rotate(-26.57)*"Slope $-\frac12$", (14,4), dir(55), lightred);
    label(rotate(-45)*"Slope $-1$", (17.5,1.5), dir(45), lightred);
    label("Slope $0$", (9.5,5), dir(90), lightred);
  \end{asy}
  \end{center}
  \caption{A plot of $\Arch_{[0,19]}(3,4)$.}
  \label{fig:arch}
\end{figure}

\chapter{Regular semi-simplicity and matching}
\label{ch:rs_matching}

\section{Regular semi-simple elements}
We first recall the notion of regularity
that first appeared in \cite[\S6]{ref:multoneconj}.

\begin{definition}
  \label{def:rs}
  Consider a $n \times n$ matrix
  \[ \begin{bmatrix} A & \uu \\ \vv^\top & d \end{bmatrix} \in \GL_n(E) \]
  in $\GL_n(E)$, where $A$ is an $(n-1) \times (n-1)$ matrix.
  Then we say this matrix is \emph{regular semi-simple} if
  \[ \left< \uu, A\uu, \dots, A^{n-2}\uu \right> \]
  and \[ \left< \vv^\top, \vv^\top A, \dots, \vv^\top A^{n-2} \right> \]
  are each a basis of $E^{n-1}$.
  Equivalently, the matrix
  \[ \left[ \vv^\top A^{i+j-2} \uu \right]_{i,j=1}^{n-1} \]
  should be nonsingular.
\end{definition}

\begin{remark}
  In \cite[Theorem 6.1]{ref:multoneconj}, this definition is shown to be equivalent to
  requiring that, under the action of conjugation by $\GL_{n-1}(E)$:
  \begin{itemize}
  \ii the matrix has trivial stabilizer; and
  \ii the $\GL_{n-1}(\ol E)$-orbit is a Zariski-closed subset of $\GL_n(\ol E)$.
  \end{itemize}
  Here $\ol E$ is as usual an algebraic closure of $E$.
\end{remark}

\begin{remark}
  [{\cite[Proposition 6.2]{ref:multoneconj}}]
  It turns out we can detect whether two regular semisimple elements
  \[
    \begin{bmatrix} A_1 & \uu_1 \\ \vv_1^\top & d_1 \end{bmatrix},
    \begin{bmatrix} A_2 & \uu_2 \\ \vv_2^\top & d_2 \end{bmatrix}
    \in \GL_{n}(E)
  \]
  are conjugate by an element of $\GL_{n-1}(E)$.
  This happens if and only if the following conditions all hold:
  \begin{itemize}
    \ii The matrices $A_1$ and $A_2$ have the same characteristic polynomial;
    \ii We have $\vv_1^\top A_1^i \uu_1 = \vv_2^\top A_2^i \uu_2$
    for every $i = 0, 1, \dots, n-2$; and
    \ii We have $d_1 = d_2$.
  \end{itemize}
  Thus, this gives a set of invariants that completely classify the orbits
  under the action of $\GL_{n-1}(E)$.

  Put another way, the invariants of
  \[ \begin{bmatrix} A & \uu \\ \vv^\top & d \end{bmatrix} \in \GL_n(E) \]
  are the (monic) characteristic polynomial of $A$
  (which has $n-1$ coefficients besides the leading coefficient),
  the values of $\vv^\top A^i \uu$ for $i = 0, \dots, n-2$
  and the number $d$ --- for a total of $2n-1$ numbers.
  \label{rem:invariants}
\end{remark}

We can now speak of regular-simplicity in each of the four
particular cases relevant to this paper.
\begin{definition}
  In the group version of the AFL:
  \begin{itemize}
    \ii We say $\gamma \in S_n(F)$ is regular semisimple
    if its image under the inclusion $S_n(F) \subseteq \GL_n(E)$ is regular semisimple.
    We write $\gamma \in S_n(F)\rs$.

    \ii For $g \in \U(\VV_n^\pm)$,
    we say $g$ is regular semisimple
    if its image under the inclusion $\U(\VV_n^\pm) \subseteq \GL_n(E)$ is regular semisimple.
    We write $g \in \U(\VV_n^\pm)\rs$.
  \end{itemize}
  In the semi-Lie version of the AFL:
  \begin{itemize}
    \ii We say $(\gamma, \uu, \vv^\top) \in S_n(F) \times V_n'$
    is regular semisimple if its image under the embedding
    \begin{equation}
      \begin{aligned}
        S_n(F) \times V_n' &\hookrightarrow \GL_{n+1}(E) \\
        (\gamma, \uu, \vv^\top) &\mapsto \begin{bmatrix} \gamma & \uu \\ \vv^\top & 0 \end{bmatrix}
      \end{aligned}
      \label{eq:embed_FJ_analytic}
    \end{equation}
    is regular semisimple.
    In other words, we require that
    both of the sets
    $\left( \uu, \gamma \uu \dots, \gamma^{n-1}\uu \right)$
    and
    $\left( \vv^\top, \vv^\top \gamma, \dots, \vv^\top \gamma^{n-1} \right)$
    are bases of $E^n$.
    In this case we write $(\gamma, \uu, \vv^\top) \in (S_n(F) \times V_n')\rs$.

    \ii For $(g,u) \in \U(\VV_n^\pm) \times \VV_n^\pm$ we say $(g, u)$
    is regular semisimple if its image under the embedding
    \begin{equation}
      \begin{aligned}
        \U(\VV_n^\pm) \times \VV_n^\pm &\hookrightarrow \GL_{n+1}(E) \\
        (g, u) &\mapsto \begin{bmatrix} g & u \\ u^\ast & 0 \end{bmatrix}
      \end{aligned}
      \label{eq:embed_FJ_geometric}
    \end{equation}
    is regular semisimple; here $u^\ast$ is the conjugate transpose.
    This is equivalent to the set $\left(  u, gu, \dots, g^{n-1}u \right)$
    being linearly independent (i.e.\ form a basis of $\VV_n^\pm$);
    in this case the independence of $\left( u^\ast, u^\ast g, \dots, u^\ast g^{n-1} \right)$
    is redundant, so it's enough to verify one condition.
    We write $(g,u) \in (\U(\VV_n^\pm) \times \VV_n^\pm)\rs$.
  \end{itemize}
  \label{def:regular}
\end{definition}

\section{Matching in the group version of the inhomogeneous AFL}
We now describe the matching condition used in the group version of AFL.
\begin{definition}
  We say $\gamma \in S_n(F)\rs$ matches the element $g \in \U(\VV_n^\pm)\rs$
  if $g$ is conjugate to $\gamma$ by an element of $\GL_{n-1}(E)$.
  By \Cref{rem:invariants}, this is an assertion that
  the invariants for $\gamma$ and $g$ coincide.
  \label{def:matching_inhomog}
\end{definition}
In that case, we have the following result.
\begin{proposition}
  [{\cite[Lemma 2.3]{ref:AFL}}; see also {\cite[(3.3.2)]{ref:AFLspherical}}]
  \label{prop:valuation_delta_matching_group}
  This definition of matching gives
  a bijection of regular semisimple orbits
  \[ [S_n(F)]\rs \xrightarrow{\sim} [\U(\VV_n^-)]\rs \amalg [\U(\VV_n^+)]\rs. \]
  Moreover, we can detect whether $\gamma \in S_n(F)\rs$ matches an orbit of
  $\U(\VV_n^-)\rs$ or $\U(\VV_n^+)\rs$ as follows.
  Suppose we write $\gamma$ in the format of \Cref{def:rs} and consider
  \[ \Delta \coloneqq \det \left[ \vv^\top A^{i+j-2} \uu \right]_{i,j=1}^{n-1} \neq 0. \]
  Then
  \begin{itemize}
    \ii $\gamma$ matches an orbit in $\U(\VV_n^-)\rs$ if $v(\Delta)$ is even;
    \ii $\gamma$ matches an orbit in $\U(\VV_n^+)\rs$ if $v(\Delta)$ is odd.
  \end{itemize}
\end{proposition}
In this paper, \Cref{conj:inhomog}
requires that $\gamma$ should match an element of $\U(\VV_n^+)\rs$
and consequently we will usually only be interested in the latter case.

\section{Matching in the semi-Lie version of the AFL}
For the semi-Lie version matching is defined analogously:
\begin{definition}
  [{\cite[\S1.3]{ref:liuFJ}}]
  We say $(\gamma, \uu, \vv^\top) \in (S_n(F) \times V_n')\rs$
  matches the element $(g, u) \in (\U(\VV_n^\pm) \times \VV_n^\pm)\rs$ if
  their images under the embeddings \eqref{eq:embed_FJ_analytic}
  and \eqref{eq:embed_FJ_geometric} are conjugate by an element of $\GL_n(E)$.
  Unwrapping this with \Cref{rem:invariants},
  an equivalent definition is to require both of the following conditions:
  \begin{itemize}
    \ii As elements of $\GL_n(E)$,
    both $g$ and $\gamma$ have the same characteristic polynomial.
    \ii We have $\vv^\top \gamma^i \uu = \left< g^i u, u \right>$ for all $0 \le i \le n-1$,
    where $\left< -,- \right>$ is the Hermitian form on $\VV_n^\pm$.
  \end{itemize}
  \label{def:matching_semi_lie}
\end{definition}
We have the following analogous criteria for matching.
\begin{proposition}
  [\cite{ref:liuFJ}]
  \label{prop:valuation_delta_matching_semilie}
  This definition of matching gives a bijection of regular semisimple orbits
  \[ [S_n(F) \times V_n']\rs \xrightarrow{\sim} [\U(\VV_n^-) \times \VV_n^-]\rs \amalg [\U(\VV_n^+) \times \VV_n^+]\rs. \]
  Moreover, we can detect whether $\guv \in S_n(F)\rs$ matches an orbit of
  $(\U(\VV_n^-) \times \VV_n^-)\rs$ or $(\U(\VV_n^+) \times \VV_n^+)\rs$ as follows:
  consider the determinant
  \[ \Delta \coloneqq \det \left[ \vv^\top \gamma^{i+j-2} \uu \right]_{i,j=1}^n \neq 0. \]
  Then
  \begin{itemize}
    \ii $\gamma$ matches an orbit in $(\U(\VV_n^-) \times \VV_n^-)\rs$ if $v(\Delta)$ is even;
    \ii $\gamma$ matches an orbit in $(\U(\VV_n^+) \times \VV_n^+)\rs$ if $v(\Delta)$ is odd.
  \end{itemize}
\end{proposition}
In this paper, \Cref{conj:semi_lie_spherical}
requires that $\guv$ should match an element of $(\U(\VV_n^+) \times \VV_n^+)\rs$
and consequently we will usually only be interested in the latter case.

\chapter{Base change}
\label{ch:satake}

This section introduces necessary background material on the base change
\[ \BC_{S_n}^{\eta^{n-1}} \colon \HH(S_n(F)) \to \HH(\U(\VV_n^+)). \]

Throughout this section we let $\Sym(n)$ denote the symmetric group in $n$ variables
with order $n!$
(since $S_n(F) \subseteq \GL_n(E)$ is already reserved for the symmetric space).

\section{Background on the Satake transformation in transformation}
We recall a general form of the Satake transformation, which will be used later.

For this subsection, $G$ will denote an arbitrary connected reductive group
over some non-Archimedean local field $F$.
We will not distinguish between $G$ and $G(F)$ when there is no confusion.

To simplify things, we will assume $G$ is unramified;
but we do \emph{not} assume $G$ is split.
Introduce the following notation:
\begin{itemize}
  \ii Let $K$ be a hyperspecial maximal compact subgroup of $G$
  (it exists because $G$ is unramified).
  \ii Let $A$ denote a maximal $F$-split torus in $G$.
  All the maximal $F$-split tori in $G$ are conjugate; let $A$ denote one of them.
  \ii Let $M$ be the centralizer of $A$; this is itself a maximal torus in $G$.
  \ii Let $\prescript{\circ}{} M \coloneqq M(F) \cap K$
  be the maximal compact subgroup of $M$.
  \ii Let $P$ denote a minimal $F$-parabolic containing $A$.
  \ii Let $\delta$ denotes the modulus character of $P$.
  It can be describes as follows.
  Let $\varpi$ denote a uniformizer for $F$ and $q$ the residue characteristic.
  Then if $\rho$ is the Weyl vector and $\mu$ is a positive cocharacter, then
  \[ \delta(\mu(\varpi)) = q^{- \left< \mu, \rho\right>}. \]
  \ii Let $N$ denote the unipotent radical of $P$.
  \ii Let $W$ be the relative Weyl group for the pair $(G,A)$,
  which acts on $\HH(M, \prescript{\circ}{} M)$.
\end{itemize}
We can now state the Satake isomorphism.
\begin{definition}
  [Satake transform]
  The \emph{Satake transform} is a canonical isomorphism of Hecke algebras
  \[ \Sat \colon \HH(G, K) \to \HH(M, \prescript{\circ}{} M)^W \]
  which is given by defining
  \[ (\Sat(f))(t) \coloneqq \delta(t)^\half \int_N f(nt) \odif n  \]
  for each $t \in M$.
\end{definition}
We are going to apply this momentarily in two situations:
once when $G$ is the general linear group (which is split),
and once when $G$ is a unitary group.


\section{The Satake transformation for the particular Hecke algebras $\HH(\GL_n(E))$ and $\HH(\U(\VV_n^+))$}
To take the Satake transform of $\HH(\U(\VV_n^+))$, we define the following abbreviations.
\begin{itemize}
  \ii Let $T$ denote the split diagonal torus of $\GL_n$.
  \ii Let
  \[ N' \coloneqq \left\{ \begin{pmatrix}
      1 & \ast & \dots & \ast \\
        & 1 & \dots & \ast \\
        &   & \ddots & \vdots \\
        &   &   & 1 \end{pmatrix}\right\} \subseteq \GL_n(E) \]
  denote the unipotent upper-triangular matrices.
\end{itemize}
Similarly for $\HH(\U(\VV_n^+))$:
\begin{itemize}
  \ii Set $m \coloneqq \left\lfloor n/2 \right\rfloor$ for brevity.
  \ii Let
  \[ A \coloneqq \left\{
    \diag(x_1, \dots, x_m, 1_{n-2m}, x_m\inv, \dots, x_1\inv) \right\} \]
  so that $A(F)$ is a maximal $F$-split torus of $\U(\VV_n^+)$.
  \ii Let $N \coloneqq N' \cap G$ denote the unipotent upper triangular matrices
  which are also unitary.
  \ii For brevity, let $W_m \coloneqq (\ZZ/2\ZZ)^m \rtimes \Sym(m)$
  be the relative Weyl group of $(G,A)$.
\end{itemize}

We can now introduce the Satake transform for our two
\emph{bona fide} Hecke algebras, using the data in Table~\ref{tab:satakestuff}.

\begin{table}[ht]
  \centering
  \begin{tabular}{lll}
    \toprule
    Group & $G' = \GL_n(E)$ & $G = \U(\VV_n^+)$ \\ \midrule
    Local field & $E$ & $F$ \\\hline
    Hyperspecial compact & $K' = \GL_n(\OO_E)$ & $K = G \cap \GL_n(\OO_E)$ \\\hline
    Max'l split torus & $T(E)$ & $A(F)$ \\\hline
    Centralizer of split torus & $T(\OO_E)$ & $A(\OO_F)$ \\\hline
    Parabolic (Borel) & Upper tri in $G'$ & Upper tri in $G$ \\\hline
    Unipotent rad.\ of parabolic & $N'$ (unipot.\ upper tri) & $N$ (unipot.\ upper tri) \\\hline
    Relative Weyl group & $\Sym(n)$ & $W_m = (\ZZ/2\ZZ)^m \rtimes \Sym(m)$ \\
    %Cocharacter group & $\ZZ^{\oplus n}$ & $\ZZ^{\oplus m}$?? \\
    %Weyl vector & $\left< \frac{n-1}{2}, \frac{n-3}{2}, \dots, -\frac{n-1}{2} \right>$
    %            & $\left< \frac{m-1}{2}, \frac{m-3}{2}, \dots, -\frac{m-1}{2} \right>$??
    \bottomrule
  \end{tabular}
  \caption{Data needed to run the Satake transformation.}
  \label{tab:satakestuff}
\end{table}

Hence, the Satake transformations obtained can be viewed as
\begin{align*}
  \Sat &\colon \HH(\GL_n(E)) \xrightarrow{\sim} \QQ[T(E) / T(\OO_E)]^{\Sym(n)} \\
  \Sat &\colon \HH(\U(\VV_n^+))\xrightarrow{\sim} \QQ[A(F) / A(\OO_F)]^{W_m}
\end{align*}
(In both cases, the modular character $\delta^{1/2}$ gives rational values,
so it is okay to work over $\QQ$.)

To make this further concrete, we remark that the cocharacter groups
involved are free abelian groups with known bases.
This identification lets us rewrite the right-hand sides above as concrete polynomials.
Specifically, we identify
\[ \QQ[T(E) / T(\OO_E)]^{\Sym(n)}
  \xrightarrow{\sim} \QQ[X_1^\pm, \dots, X_n^\pm]^{\Sym(n)} \]
by identifying $X_i$ with the
cocharacter corresponding to injection into the $i$\ts{th} factor.
Similarly, we identify
\[ \QQ[A(F) / A(\OO_F)]^{W_m}
  \xrightarrow{\sim} \QQ[Y_1^{\pm}, \dots, Y_m^{\pm}]^{W_m} \]
by identifying $Y_i + Y_i^{-1}$
with the cocharacter corresponding to
\[ x \mapsto \diag(1, \dots, x, \dots, x\inv, \dots, 1) \]
where $x$ is in the $i$\ts{th} position and $x\inv$ is in the $(n-i)$\ts{th} position,
and all other positions are $1$.
Here $\QQ[Y_1^{\pm}, \dots, Y_m^{\pm}]^{W_m}$
denotes the ring of symmetric polynomials in $Y_i + Y_i^{-1}$.

So, henceforth, we will consider
\begin{align*}
  \Sat &\colon \HH(\GL_n(E)) \xrightarrow{\sim} \QQ[X_1^\pm, \dots, X_n^\pm]^{\Sym(n)} \\
  \Sat &\colon \HH(\U(\VV_n^+)) \xrightarrow{\sim} \QQ[Y_1^{\pm}, \dots, Y_m^{\pm}]^{W_m}.
\end{align*}

\section{Relation of Satake transformation to base change}
Let
\[ \BC \colon \HH(\GL_n(E)) \to \HH(\U(\VV_n^+)) \]
denote the stable base change morphism from $\GL_n(E)$ to the unitary group $\U$.
The relevance of the Satake transformation is that
(see e.g.\ \cite[Proposition 3.4]{ref:leslie})
it gives a way to make this $\BC$ completely explicit:
we have a commutative diagram
\begin{center}
\begin{tikzcd}
  \HH(\GL_n(E))  \ar[r, "\sim"', "\Sat"] \ar[d, "\BC"]
    & \QQ[X_1^\pm, \dots, X_n^\pm]^{\Sym(n)} \ar[d, "\BC"] \\
  \HH(\U(\VV_n^+)) \ar[r, "\sim"', "\Sat"]
    & \QQ[Y_1^\pm, \dots, Y_m^\pm]^{W_m}
\end{tikzcd}
\end{center}
Here the right arrow is also denoted $\BC$ following \cite{ref:AFLspherical}
(although it is denoted $\nu$ in \cite{ref:leslie}).
This gives a way in which we can concretely calculate the map $\BC$
in some situations.

\section{The map $\BC^{\eta^{n-1}}_{S_n}$}
Before we can define the map $\BC^{\eta^{n-1}}_{S_n}$
we need one more piece of notation.
Consider the following map.
\begin{definition}
  [$\rproj$]
  Denote by $\rproj \colon \GL_n(E) \surjto S_n(F)$ the projection defined by
  \[ \rproj(g) \coloneqq g \bar{g}\inv. \]
\end{definition}
Then $\rproj$ induces a map
\begin{align*}
  \rproj_\ast \colon \HH(\GL_n(E)) &\to \HH(S_n(F)) \\
  \rproj_\ast(f')\left( g\bar{g}\inv \right) &= \int_{\GL_n(F)} f'(gh) \odif h
\end{align*}
by integration on the fibers.
A similar twisted version by $\eta$
\begin{align*}
  \rproj_\ast^\eta \colon \HH(\GL_n(E)) &\to \HH(S_n(F)) \\
  \rproj_\ast^\eta(f')\left( g\bar{g}\inv \right) &= \int_{\GL_n(F)} f'(gh) \eta(gh) \odif h
\end{align*}
is defined analogously,
where as before $\eta(g) = (-1)^{v(\det g)}$ in a slight abuse of notation.

Then Leslie \cite{ref:leslie} shows the following result.
\begin{theorem}
  [{\cite[Theorem 3.2 and Proposition 3.4]{ref:leslie}}]
  Both maps $\rproj_\ast$ and $\rproj_\ast^\eta$ induce isomorphisms
  \begin{align*}
    \BC_{S_n} \colon \HH(S_n(F)) &\xrightarrow{\sim} \HH(\GL_n(E)) \\
    \BC^{\eta^{n-1}}_{S_n} \colon \HH(S_n(F)) &\xrightarrow{\sim} \HH(\GL_n(E))
  \end{align*}
  such that
  \begin{align*}
    \BC &= \BC_{S_n} \circ \rproj_\ast \\
    \BC &= \BC^{\eta^{n-1}}_{S_n} \circ \rproj_\ast^{\eta^{n-1}}.
  \end{align*}
\end{theorem}
We take these isomorphisms promised by this theorem
as the definition of $\BC_{S_n}$ and $\BC^{\eta^{n-1}}_{S_n}$ in our conjectures
(noting when $n$ is odd they coincide, as $\eta^{n-1} = 1$).

When combined with the Satake information we have, we get the following diagram.
\begin{center}
\begin{tikzcd}
  \HH(\GL_n(E)) \ar[dd, "\rproj_\ast^{\eta^{n-1}}"', bend right = 60] \ar[r, "\sim"', "\Sat"] \ar[d, "\BC"]
    & \QQ[X_1^\pm, \dots, X_n^\pm]^{\Sym(n)} \ar[d, "\BC"] \\
  \HH(\U(\VV_n^+)) \ar[r, "\sim"', "\Sat"]
    & \QQ[Y_1^\pm, \dots, Y_m^\pm]^{W_m} \\
    \HH(S_n(F)) \ar[u, "\sim", "\BC_{S_n}^{\eta^{n-1}}"']
\end{tikzcd}
\end{center}

\section{Calculation of $\BC_{S_n}$ when $n = 3$}
The goal of this section is to make the base change
fully known in the special case $n = 3$, where $m = \left\lfloor n/2 \right\rfloor = 1$.
(In this case $\BC_{S_n}^{\eta^{n-1}} = \BC_{S_n}$ as $\eta^2 = 1$.)
The completed result is \Cref{prop:BC_S3}.

This calculation parallels the $n = 2$ case that was done in \cite[Lemma 7.1.1]{ref:AFLspherical}.
However, we will not use these results again later on.
When it is not more difficult, some of the results will be stated for all $n$,
rather than $n = 3$ specifically.

\subsection{Overview}
Throughout this subsection, we use the shorthand
\[ \varpi^{(n_1, n_2, n_3)} \coloneqq \diag(\varpi^{n_1}, \varpi^{n_2}, \varpi^{n_3}). \]
As a $\QQ$-module, the spaces $\HH(\U(\VV_n^+))$ and $\HH(S_n(F))$
have a canonical basis of indicator functions indexed by $\ZZ$:
\begin{itemize}
  \ii $\HH(S_n(F))$ has $\QQ$-module basis $\mathbf{1}_{K'_{S,j}}$ for $j \ge 0$.
  \ii $\HH(\U(\VV_n^+))$ has a $\QQ$-module basis given by the indicator functions
  \[ \mathbf{1}_{\varpi^{-r} \Mat(\OO_E) \cap \U(\VV_n^+)} \]
  for $r \ge 0$.
\end{itemize}
On the other hand, the natural $\QQ$-module basis for $\HH(\GL_n(E))$, namely
\[ \mathbf{1}_{K'\varpi^{(n_1, n_2, n_3)}K'} \]
is given by triples of integers $n_1 \ge n_2 \ge n_3 \ge 0$, and is much larger.
So explicit calculations for the $\rproj_\ast$ or the Satake transforms viewed in
$\CC[X_1, X_2, X_3]^{\Sym(n)}$ are nontrivial if one works with the entire basis.

Hence the overall strategy, to reduce the amount of work we have to do,
is to focus on only the $\ZZ$-indexed elements
\[
  \mathbf{1}_{\Mat_3(\OO_E), v\circ\det=r}
  = \sum_{\substack{n_1 \ge n_2 \ge n_3 \\ n_1 + n_2 + n_3 = r}}
  \mathbf{1}_{K'\varpi^{(n_1, n_2, n_3)}K'} \in \HH(\GL_n(E))
\]
for $r \ge 0$.
This aggregated indicator function is easier to compute,
because given an explicit matrix it is somewhat easier
to evaluate \[ \mathbf{1}_{\Mat_3(\OO_E), v\circ\det=r} \]
at it (one only needs to check it has $\OO_E$ entries
and that the determinant has valuation $r$,
rather than determining the exact coset $K'\varpi^{(n_1, n_2, n_3)}K'$).

\subsection{Satake transform of the determinant characteristic function on the top arrow}
This is the easiest calculation, and we do it for all $n$ rather than just $n = 3$.
\begin{proposition}
  [Satake transform for $v \circ \det = r$]
  For every integer $r \ge 0$, we have
  \[ \Sat(\mathbf{1}_{\Mat_n(\OO_E), v\circ\det=r})
    = q^{(n-1)r} \sum_{e_1 \dots + e_n = r} X_1^{e_1} \dots X_n^{e_n}. \]
\end{proposition}
\begin{proof}
  We evaluate the coefficient $X_1^{e_1} \dots X_n^{e_n}$.
  Choose a cocharacter $\mu$,
  and suppose $\mu(\varpi) = \varpi^{(e_1, \dots, e_n)}$ with $n_1 \ge n_2 \ge n_3$.
  Let $q_E = q^2$ be the residue characteristic of $E$.
  Take the upper triangular matrices as our Borel subgroup as usual,
  so the unipotent radical of this Borel subgroup
  are the unipotent upper triangulars $N'$ which we describe as
  \[ N' \coloneqq \left\{
      \begin{pmatrix}
      1 & y_{12} & y_{13} & \dots & y_{1n} \\
        & 1 & y_{23} & \dots & y_{2n} \\
        &   & 1 & \dots & y_{3n} \\
        &   &   & \ddots & \vdots  \\
        &   &   &   & 1
      \end{pmatrix}
    \mid y_{12}, \dots, y_{(n-1)n}\in E \right\} \]
  and with additive Haar measure is $\odif{y_{12}, y_{23} \dotso, y_{(n-1)n}}$.
  Recall also the Weyl vector for $\GL_n(E)$ is just
  \[ \rho_{\GL_n(E)} = \left< \frac{n-1}{2}, \frac{n-3}{2}, \dots, -\frac{n-1}{2} \right>. \]
  Compute
  \begin{align*}
    &\Sat(\mathbf{1}_{\Mat_n(\OO_E), v\circ\det=r})(\mu(\varpi)) \\
    &= \delta(\mu(\varpi))^\half \int_{n' \in N'}
      \mathbf{1}_{\Mat_n(\OO_E, v\circ \det = r)} (\mu(\varpi) n') \odif{n'} \\
    &= q_E^{-\left< \mu, \rho\right>}
    \underbrace{\int_{y_{12} \in E} \int_{y_{13} \in E} \dotso \int_{y_{(n-1)n} \in E}}_{\binom n2 \text{ integrals}} \\
    &\qquad
      \mathbf{1}_{\Mat_3(\OO_E), v \circ \det = r}
      \left( \begin{pmatrix}
        \varpi^{e_1} & \varpi^{e_1} y_{12} & \varpi^{e_1} y_{13} & \dots & \varpi^{e_1} y_{1n} \\
        & \varpi^{e_2} & \varpi^{e_2} y_{23} & \dots & \varpi^{e_2} y_{2n} \\
        &   & \varpi^{e_3} & \dots & \varpi^{e_3} y_{3n} \\
        &   &   & \ddots & \vdots  \\
        &   &   &   & \varpi^{e_n}
        \end{pmatrix} \right) \\
    &\qquad \odif{y_{12}, y_{23} \dotso, y_{(n-1)n}} \\
    &= q_E^{-\left(\frac{n-1}{2}e_1 + \frac{n-3}{2}e_2 + \dots + -\frac{n-1}{2} e_n \right)}
    \mathbf{1}_{e_1 + \dots + e_n = r}
    \underbrace{\int_{y_{12} \in E} \int_{y_{13} \in E} \dotso \int_{y_{(n-1)n} \in E}}_{\binom n2 \text{ integrals}} \\
    &\qquad \prod_{1 \le i < j \le n} \mathbf{1}_{\OO_E}(\varpi^{e_i} y_{ij}) \odif{y_{ij}} \\
    &= q_E^{-\left(\frac{n-1}{2}e_1 + \frac{n-3}{2}e_2 + \dots + -\frac{n-1}{2} e_n \right)}
    \mathbf{1}_{e_1 + \dots + e_n = r} \prod_{1 \le i < j \le n} q_E^{e_i} \\
    &= q_E^{-\left(\frac{n-1}{2}e_1 + \frac{n-3}{2}e_2 + \dots + -\frac{n-1}{2} e_n \right)}
    \mathbf{1}_{e_1 + \dots + e_n = r} \prod_{1 \le i \le n} q_E^{(n-i)e_i} \\
    &= \mathbf{1}_{e_1 + \dots + e_n = r} \prod_{1 \le i \le n}^n q_E^{\frac{n-1}{2} e_i} \\
    &= q_E^{\frac{n-1}{2} r} \mathbf{1}_{e_1 + \dots + e_n = r} \\
    &= \begin{cases}
      q^{\frac{n-1}{2} r} & \text{if } e_1 + \dots + e_n = r \\
      0 & \text{otherwise}.
    \end{cases}
  \end{align*}
  This gives the sum claimed earlier.
\end{proof}


\subsection{Satake transform of the indicator on the bottom arrow}
\begin{proposition}
  [Satake transform for $\varpi^{-r} \Mat_3(\OO_E) \cap \U(\VV_3^+)$]
  For each $r \ge 0$ we have
  \[ \Sat\left(\mathbf{1}_{\varpi^{-r} \Mat_3(\OO_E) \cap \U(\VV_3^+)}\right)
    = \sum_{i=0}^r q^{2r - \mathbf{1}_{r \equiv i \bmod 2}} Y_1^{\pm i} \]
  where we adopt the shorthand
  \[
    Y_1^{\pm i} \coloneqq
    \begin{cases}
      Y_1^i + Y_1^{-i} & i > 0 \\
      1 & i = 0 .
    \end{cases}
  \]
\end{proposition}
\begin{proof}
  We first need to describe \[ N = N' \cap \U(\VV_3^+) \] a little more carefully.
  For $n \in N'$ we have
  \[
    n^\ast \beta n
    =
    \begin{pmatrix} 1 \\ \bar{y_1} & 1 \\ \bar{y_2} & \bar{y_3} & 1 \end{pmatrix}
    \beta
    \begin{pmatrix}
      1 & y_1 & y_2 \\
        & 1 & y_3 \\
        & & 1
    \end{pmatrix}
    = \begin{pmatrix}
      & & 1 \\
      & 1 & y_3 + \bar{y_1} \\
      1 & y_1 + \bar{y_3} & y_2 + \bar{y_2} + y_3 \bar{y_3}
    \end{pmatrix}.
  \]
  So $n \in N$ if and only if the above matrix equals $\beta$, which means
  \[ 0 = y_3 + \bar{y_1} = y_2 + \bar{y_2} + y_3 \bar{y_3}. \]
  Then we can re-parametrize by $z_1, z_2, z_3 \in F$ according to
  \begin{align*}
    y_3 &= z_1 + z_2 \sqrt\eps \\
    y_2 &= -\frac{z_1^2 + z_2^2 \eps}{2} + z_3\sqrt\eps \\
    y_1 &= -z_1 + z_2\sqrt\eps.
  \end{align*}
  Back to the original task.
  For each $i \ge 0$ we can evaluate the Satake transform at the element
  $\nu(\varpi) = \diag(\varpi^i, 1, \varpi^{-i})$, for the cocharacter $\nu$
  corresponding to $Y_1^i + Y_1^{-i}$:
  \begin{align*}
    &\Sat\left( \mathbf{1}_{\varpi^{-r} \Mat_3(\OO_E) \cap \U(\VV_3^+)}\right)
      \left( \nu(\varpi)  \right) \\
    &= \delta(\nu(\varpi))^\half \int_{n \in N}
      \mathbf{1}_{\varpi^{-r} \Mat_3(\OO_E) \cap \U(\VV_3^+)}
      \left( \nu(\varpi) n' \right) \odif n \\
    &= \delta(\nu(\varpi))^\half \int_{n \in N}
      \mathbf{1}_{{\varpi^{-r}} \Mat_3(\OO_E) \cap \U(\VV_3^+)}
      \left( \begin{pmatrix} \varpi^i & \varpi^i y_1 & \varpi^i y_2 \\
               & 1 & y_3 \\
               & & \varpi^{-i} \end{pmatrix} \right) \odif n
  \end{align*}
  The matrix itself is always in $\U(\VV_3^+)$, because it's the product of two unitary matrices.
  So the indicator needs to check whether all the entries have valuation at least $-r$.
  If we switch characterization to the coordinates $z_1$, $z_2$, $z_3$ we described earlier,
  we see that the conditions are
  \begin{align*}
    i &\le r, \\
    v(z_1) &\ge -r, \\
    v(z_2) &\ge -r,\\
    v(z_3) &\ge -(r+i),\\
    v(z_1^2 + z_2^2 \eps) &\ge -(r+i).
  \end{align*}
  Assume $i \le r$ henceforth.
  The condition for $z_1$ and $z_2$ then really says
  \[ \min(v(z_1), v(z_2)) \ge -\left\lfloor \frac{r+i}{2} \right\rfloor. \]
  So the integral factors as a triple integral
  \[
    \int_{z_1 \in F}
    \int_{z_2 \in F}
    \int_{z_3 \in F}
    \mathbf{1}_{\varpi^{-\left\lfloor \frac{r+i}{2} \right\rfloor} \OO_F}(z_1)
    \mathbf{1}_{\varpi^{-\left\lfloor \frac{r+i}{2} \right\rfloor} \OO_F}(z_2)
    \mathbf{1}_{\varpi^{-(r+i)} \OO_F}(z_3)
    \odif{z_1,z_2,z_3}
  \]
  which is equal to
  \[ q^{2\left\lfloor \frac{r+i}{2} \right\rfloor+r+i}. \]
  Meanwhile, $\delta(\nu(\varpi))^{\half} = q^{-2i}$.
  In summary,
  \[
    \Sat\left( \mathbf{1}_{\varpi^{-r} \Mat_3(\OO_E) \cap \U(\VV_3^+)}\right) \left( \nu(\varpi) \right)
    =
    \begin{cases}
      q^{2\left\lfloor \frac{r+i}{2} \right\rfloor - i + r} & i \le r \\
      0 & i > r
    \end{cases}
  \]
  Finally, since
  \[ 2\left\lfloor \frac{r+i}{2} \right\rfloor - i + r
    = \begin{cases}
      2r & r+i \text{ is even} \\
      2r-1 & r+i \text{ is odd}
    \end{cases}
  \]
  we get the formula claimed.
\end{proof}

\subsection{Integration over fiber}
\begin{proposition}
  [Integration over fiber]
  For every integer $r \ge 0$, we have
  \begin{align*}
    &\rproj_\ast(\mathbf{1}_{\Mat_3(\OO_E), v\circ\det=r}) \\
    &= \sum_{j=0}^r \left(
      \sum_{i=0}^{2(r-j)} \min \left( 1 + \left\lfloor \frac i2 \right\rfloor,
        1 + \left\lfloor \frac{2(r-j)-i}{2} \right\rfloor \right) q^i \right)
        \mathbf{1}_{K'_{S,j}}.
  \end{align*}
\end{proposition}
\begin{proof}
  The coefficient of $\mathbf{1}_{K'_{S,j}}$ will be equal to
  the evaluation of the integral at any $g$ such that $g\bar{g} \in K'_{S,j}$.
  Fixing $j \ge 0$, we are going to take the choice
  \[
    g = \begin{pmatrix}
      1 &   & \varpi^{-j} \sqrt{\eps} \\
      & 1 \\
      &   & 1
    \end{pmatrix}.
  \]
  We need to check this choice of $g$ indeed satisfies $g\bar{g}\inv \in K'_{S,j}$.
  This follows as
  \[ \bar{g} = \begin{pmatrix} 1 &   & -\varpi^{-j} \sqrt{\eps} \\ & 1 \\ &   & 1 \end{pmatrix}
    \implies \bar{g}\inv = \begin{pmatrix} 1 &   & \varpi^{-j} \sqrt{\eps} \\ & 1 \\ &   & 1 \end{pmatrix}
  \]
  and therefore
  \[
    g\bar{g}\inv = \begin{pmatrix}
      1 &   & 2\varpi^{-j} \sqrt{\eps} \\
      & 1 \\
      &   & 1
    \end{pmatrix} \in K'_{S,j}
  \]
  as needed.

  Having chosen the representative $g$, we aim to calculate the right-hand side of
  \[
    \rproj_\ast(\mathbf{1}_{\Mat_3(\OO_E), v\circ\det=r})(g\bar{g})
    = \int_{h \in \GL_3(F)} \mathbf{1}_{\Mat_3(\OO_E), v\circ\det=r}(gh) \odif h.
  \]
  We take (non-Archimedean) Iwasawa decomposition of $h \in \GL_3(F)$ to rewrite it as
  \[
    h =
    \begin{pmatrix} x_1 \\ & x_2 \\ && x_3 \end{pmatrix}
    \begin{pmatrix} 1 & y_1 & y_2 \\ & 1 & y_3 \\ & & 1 \end{pmatrix}
    k
  \]
  for $k \in \GL_3(\OO_F) \subseteq K'$, which does not affect the indicator function.
  Here $x_1, x_2, x_3 \in F^\times$ and $y_1, y_2, y_3 \in F$.
  In that case, note that
  \begin{align*}
    gh
    &=
    \begin{pmatrix}
      1 &   & \varpi^{-j}\sqrt\eps \\
      & 1 \\
      &   & 1
    \end{pmatrix}
    \begin{pmatrix} x_1 \\ & x_2 \\ && x_3 \end{pmatrix}
    \begin{pmatrix} 1 & y_1 & y_2 \\ & 1 & y_3 \\ & & 1 \end{pmatrix} k \\
    &=
    \begin{pmatrix}
      1 &   & \varpi^{-j}\sqrt\eps \\
      & 1 \\
      &   & 1
    \end{pmatrix}
    \begin{pmatrix} x_1 & x_1 y_1 & x_1 y_2 \\ & x_2 & x_2 y_3 \\ & & x_3 \end{pmatrix} k \\
    &=
    \begin{pmatrix}
      x_1 & x_1 y_1 & x_1 y_2 + x_3 \varpi^{-j} \sqrt\eps \\
      & x_2 & x_2 y_3 \\
      & & x_3
    \end{pmatrix}
    k.
  \end{align*}
  Hence, we can rewrite the $\rproj_\ast(\mathbf{1}_{\Mat_3(\OO_E), v\circ\det=r})$
  as a six-fold integral
  \begin{align*}
    &\rproj_\ast(\mathbf{1}_{\Mat_3(\OO_E), v\circ\det=r}) \\
    &= \int_{x_1 \in F^\times} \int_{x_2 \in F^\times} \int_{x_3 \in F^\times}
    \int_{y_1 \in F} \int_{y_2 \in F} \int_{y_3 \in F} \\
    &\quad \mathbf{1}_{\Mat_3(\OO_E), v\circ\det=r} \left(
    \begin{pmatrix}
      x_1 & x_1 y_1 & x_1 y_2 + x_3 \varpi^{-j} \sqrt\eps \\
      & x_2 & x_2 y_3 \\
      & & x_3
    \end{pmatrix}
    \right) \\
    &\quad \odif[{\times,\times,\times}]{x_1,x_2,x_3,y_1,y_2,y_3}.
  \end{align*}
  Apparently the indicator function only depends on the valuations,
  so accordingly we rewrite the six-fold integral as a discrete sum over the valuations
  $\alpha_i \coloneqq v(x_i)$.
  Then the conditions are that
  \begin{align*}
    &\alpha_1 \ge 0, \quad \alpha_2 \ge 0, \quad \alpha_3 \ge j \\
    &v(y_1) \ge - \alpha_1, \quad v(y_2) \ge - \alpha_1, \quad v(y_3) \ge -\alpha_2.
  \end{align*}
  We have $\Vol(\varpi^{\alpha_i} \OO_F^\times) = 1$
  and $\Vol(\varpi^{-\alpha_i} \OO_F) = q^{\alpha_i}$.
  Hence the integral can be rewritten as the discrete sum
  \begin{align*}
    \sum_{\substack{\alpha_1 + \alpha_2 + \alpha_3 = r \\ \alpha_1 \ge 0 \\ \alpha_2 \ge 0 \\ \alpha_3 \ge j}}
    q^{\alpha_1} \cdot q^{\alpha_1} \cdot q^{\alpha_2}
    &= \sum_{\substack{\alpha_1 + \alpha_2 \le r-j \\ \alpha_1 \ge 0 \\ \alpha_2 \ge 0}}
    q^{2\alpha_1+\alpha_2} \\
    &= \sum_{i=0}^{2(r-j)}
    \min \left( 1 + \left\lfloor \frac i2 \right\rfloor,
      1 + \left\lfloor \frac{2(r-j)-i}{2} \right\rfloor
    \right) q^i
  \end{align*}
  as desired.
\end{proof}

\subsection{Base change from $\HH(\U(\VV_3^+))$ to $\HH(S_3(F))$}
We first need to determine an element of $\HH(\U(\VV_n^+))$
which is in the pre-image of
\[ \mathbf{1}_{\varpi^{-r} \Mat_3(\OO_E) \cap \U(\VV_3^+)} \]
under $\BC \colon \HH(\GL_3(E)) \to \HH(\U(\VV_3^+))$.

For convenience, we define the shorthand
\[
  \HH(\GL_3(E)) \ni
  f'_r \coloneqq \begin{cases}
    \mathbf{1}_{\Mat_3(\OO_E), v \circ \det = r} & r \ge 0 \\
    0 & r < 0
  \end{cases}
\]
for every integer $r$.
We start with the following intermediate calculation.
\begin{align*}
  &\BC\left( \Sat \left( f'_r - q^2 f'_{r-1} \right) \right) \\
  &= \BC \left(
    q^{2r} \sum_{n_1+n_2+n_3=r} X_1^{n_1} X_2^{n_2} X_3^{n_3}
    - q^2 \cdot q^{2(r-1)} \sum_{n_1+n_2+n_3=(r-1)} X_1^{n_1} X_2^{n_2} X_3^{n_3} \right) \\
  &= q^{2r} \left( \sum_{n_1+n_2+n_3=r} Y_1^{n_1-n_3} - \sum_{n_1+n_2+n_3=(r-1)} Y_1^{n_1-n_3} \right) \\
  &= q^{2r} \left( \sum_{n_1+n_3=r} Y_1^{n_1-n_3} \right) \\
  &= q^{2r} \left( Y_1^{r} + Y_1^{r-2} + \dots + Y_1^{-r} \right).
  \intertext{Replacing $r$ with $r-1$ gives}
  &\BC\left( \Sat \left(f'_{r-1} - q^2 f'_{r-2} \right) \right) \\
  &= q^{2r-2} \left( Y_1^{r-1} + Y_1^{r-3} + \dots + Y_1^{-(r-1)} \right).
  \intertext{Adding the former equation to $q$ times the latter gives}
  &\BC\left( \Sat\left( f'_r + (q-q^2) f'_{r-1} - q^3 f'_{r-2} \right) \right) \\
  &= q^{2r} \left( Y_1^r + Y_1^{r-2} + \dots + Y_1^{-r} \right)
  + q^{2r-1} \left( Y_1^{r-1} + Y_1^{r-3} + \dots + Y_1^{-(r-1)} \right) \\
  &= \Sat(\mathbf{1}_{\varpi^{-r} \Mat_3(\OO_E) \cap \U(\VV_3^+)}).
\end{align*}
This shows that
\[ \BC(f'_r + (q-q^2) f'_{r-1} - q^3 f'_{r-2}) =
  \mathbf{1}_{\varpi^{-r} \Mat_3(\OO_E) \cap \U(\VV_3^+)} \]
so indeed $f'_r + (q-q^2) f'_{r-1} - q^3 f'_{r-2}$
lies in the desired pre-image of the map $\BC \colon \HH(\GL_3(E)) \to \HH(\U(\VV_3^+))$.

On the other hand, it is easy to check that
\begin{align*}
  &\rproj_\ast(f'_r -q^2 f'_{r-1}) \\
  &= \sum_{j=0}^r \Bigg[
      \sum_{i=0}^{2(r-j)} \min \left( 1 + \left\lfloor \frac i2 \right\rfloor,
      1 + \left\lfloor \frac{2(r-j)-i}{2} \right\rfloor \right) q^i \\
  &\qquad - \sum_{i=0}^{2(r-1-j)} \min \left( 1 + \left\lfloor \frac i2 \right\rfloor,
    1 + \left\lfloor \frac{2((r-1)-j)-i}{2} \right\rfloor \right) q^{i+2}
  \Bigg] \mathbf{1}_{K'_{S,j}} \\
  &= \sum_{j=0}^r \left[ 1+q+q^2+\dots+q^{r-j} \right] \mathbf{1}_{K'_{S,j}} \\
  \intertext{so}
  &\rproj_\ast(f'_r -q^2 f'_{r-1} + q \left( f'_{r-1} - q^2 f'_{r-3} \right)) \\
  &= \sum_{j=0}^r \left[ (1+q+q^2+\dots+q^{r-j})+(q+q^2+\dots+q^{r-j}) \right] \mathbf{1}_{K'_{S,j}} \\
  &= \sum_{j=0}^r \left[ 1 + 2q + 2q^2 + \dots + 2q^{r-j} \right] \mathbf{1}_{K'_{S,j}}.
\end{align*}
To summarize, the completed commutative diagram can be written in full as
\begin{center}
\begin{tikzcd}
  \begin{tabular}{c} $f'_r + (q-q^2) f'_{r-1}$ \\ $- q^3 f'_{r-2} \in \HH(\GL_3(E))$ \end{tabular}
    \ar[dd, "\rproj_\ast"', mapsto, bend right = 50]
    \ar[r, "\Sat", mapsto] \ar[d, "\BC", mapsto]
    & \dots \in \QQ[X_1^\pm, X_2^\pm, X_3^\pm]^{\Sym(3)} \ar[d, "\BC", mapsto] \\
  \begin{tabular}{c} $\mathbf{1}_{\varpi^{-r} \Mat_3(\OO_E) \cap \U(\VV_3^+)}$ \\ $\in \HH(\U(\VV_3^+))$ \end{tabular}
    \ar[r, "\Sat", mapsto]
    & \begin{tabular}{l}
      $q^{2r} \left( Y_1^{\pm r} + \dotsb + Y_1^{\mp r} \right)$ \\
      $+ q^{2r-1} \left( Y_1^{\pm(r-1)} + \dotsb + Y_1^{\mp(r-1)} \right)$ \\
      $\in \QQ[Y_1^\pm]^{W_1}$
      \end{tabular} \\
  \begin{tabular}{c}
    $\sum_{j=0}^r \big[ 1 + 2q + 2q^2$ \\
    $+ \dots + 2q^{r-j} \big] \mathbf{1}_{K'_{S,j}}$ \\
    $\in \HH(S_3(F))$
  \end{tabular} \ar[u, "\sim", "\BC_{S_3}"', mapsto]
\end{tikzcd}
\end{center}

Thus, we arrive at the following:
\begin{proposition}
  [Base change $\BC_{S_3}$]
  \label{prop:BC_S3}
  For $n = 3$, we have
  \begin{align*}
    \BC_{S_3} \left( \sum_{j=0}^r \left[ 1 + 2q + 2q^2 + \dots + 2q^{r-j} \right]
    \mathbf{1}_{K'_{S,j}} \right)
    &= \mathbf{1}_{\varpi^{-r} \Mat_3(\OO_E) \cap \U(\VV_3^+)} \\
    \BC_{S_3} \left( \mathbf{1}_{K'_{S,r}}
    + \sum_{j=0}^{r-1} 2q^{r-j} \mathbf{1}_{K'_{S,j}} \right)
    &= \mathbf{1}_{K\varpi^{(r,0,-r)}K}
  \end{align*}
  for every integer $r \ge 0$.
\end{proposition}
\begin{proof}
  The first equation is the one we just proved.
  The second one follows by noting that
  \[
    \mathbf{1}_{K\varpi^{(r,0,-r)}K}
    = \mathbf{1}_{\varpi^{-r} \Mat_3(\OO_E) \cap \U(\VV_3^+)}
    - \mathbf{1}_{\varpi^{-(r-1)} \Mat_3(\OO_E) \cap \U(\VV_3^+)}
  \]
  so one merely subtracts the left-hand sides evaluated at $r$ and $r-1$ for $r \ge 1$
  to get
  \begin{align*}
    &\phantom= \sum_{j=0}^r \left[ 1 + 2q + 2q^2 + \dots + 2q^{r-j} \right] \mathbf{1}_{K'_{S,j}}
    - \sum_{j=0}^{r-1} \left[ 1 + 2q + 2q^2 + \dots + 2q^{(r-1)-j} \right] \mathbf{1}_{K'_{S,j}} \\
    &= \mathbf{1}_{K'_{S,r}} +
      \sum_{j=0}^{r-1} \left[ 1 + 2q + 2q^2 + \dots + 2q^{r-j} \right] \mathbf{1}_{K'_{S,j}}
    - \sum_{j=0}^{r-1} \left[ 1 + 2q + 2q^2 + \dots + 2q^{(r-1)-j} \right] \mathbf{1}_{K'_{S,j}} \\
    &= \mathbf{1}_{K'_{S,r}} + \sum_{j=0}^{r-1} \left[ 2q^{r-j}\mathbf{1}_{K'_{S,j}} \right].
  \end{align*}
  as claimed.
\end{proof}


\chapter{Synopsis of the orbital integral $\Orb(\gamma, \phi, s)$ for $\gamma \in S_3(F)\rs$ and $\phi \in \HH(S_3(F), K')$}
\label{ch:orbital0}

This section defines the orbital integral
and describes the parameters which we will use to express our answer.

\section{Initial definition of the orbital integral for general $S_n(F)$}
Let $H = \GL_{n-1}(F)$.
Then $H$ has a natural embedding into $G = \GL_n(E)$ by
\[ h \mapsto \begin{bmatrix} h & 0 \\ 0 & 1 \end{bmatrix} \]
which endows it with an action $S_n(E)$.
Then our orbital integral is defined as follows.
\begin{definition}
  For brevity let $\eta(h) \coloneqq \eta(\det h)$ for $h \in H$.
  For $\gamma \in S_n(F)$, $\phi \in \HH(S_n(F), K')$, and $s \in \CC$,
  we define the orbital integral by
  \[ \Orb(\gamma, \phi, s) \coloneqq
    \int_{h \in H} \phi(h\inv \gamma h) \eta(h)
    \left\lvert \det(h) \right\rvert_F^{-s} \odif h. \]
  \label{def:orbital0}
\end{definition}

\section{Basis for the indicator functions in $\HH(S_3(F), K')$}
\label{ch:orbital0_hecke_basis}
From now on assume $n = 3$.
We have the symmetric space
\[ S_3(F) \coloneqq \left\{ g \in \GL_3(E) \mid g \bar{g} = \id_3 \right\}. \]
which has a left action under $\GL_3(E)$ by $g \cdot s \mapsto gs\bar{g}\inv$.

Then $S_3(F)$ admits the following decomposition, which we will use:
\begin{lemma}
  [Cartan decomposition of $S_3(F)$]
  For each integer $r \ge 0$ let
  \[ K'_{S,r} \coloneqq \GL_3(\OO_E) \cdot \begin{bmatrix} 0 & 0 & \varpi^r \\ 0 & 1 & 0 \\ \varpi^{-r} & 0 & 0 \end{bmatrix} \]
  denote the orbit of
  $\begin{bmatrix} 0 & 0 & \varpi^r \\ 0 & 1 & 0 \\ \varpi^{-r} & 0 & 0 \end{bmatrix}$
  under the left action of $\GL_3(\OO_E)$.
  Then we have a decomposition
  \[ S_3(F) = \coprod_{r \geq 0} K'_{S,r}. \]
\end{lemma}
The $r=0$ case will be given a special shorthand,
and can be expressed in a few equivalent ways:
\begin{align*}
  K'_S
  &\coloneqq K'_{S,0} \\
  &= \GL_3(\OO_E) \cdot \begin{bmatrix} & & 1 \\ & 1 \\ 1 \end{bmatrix} \\
  &= \GL_3(\OO_E) \cdot \id_3 = S_3(F) \cap \GL_3(\OO_E).
\end{align*}
One can equivalently define $K'_{S,r}$ to be the part of $S_3(F)$
for which the most negative valuation among the nine entries is $-r$.

For $r \geq 0$, define
\[ K'_{S, \le r} \coloneqq S_3(F) \cap \varpi^{-r} \GL_3(\OO_E). \]
We can re-parametrize the problem according to the following.
\begin{proposition}
  \[ K'_{S, \le r} = K'_{S,0} \sqcup K'_{S,1} \sqcup \dots \sqcup K'_{S,r}. \]
\end{proposition}
Then an integral over each $K'_{S, \le r}$ lets us extract the integrals over $K'_{S,r}$.
\begin{proposition}
  For $r \ge 0$, the indicator functions $\mathbf{1}_{K'_{S, \le r}}$
  form a basis of $\HH(S_3(F), K')$.
\end{proposition}
\todo{this is like an $\HH(G', K')$-basis I think? double check this}

Then, our goal is to compute for
\begin{equation}
  \pdv{}{s}\Orb(\gamma, \mathbf{1}_{K'_{S, \le r}}, s)
  \label{eq:orbital_goal}
\end{equation}
at $s=0$ for any $r > 0$ as well.
Note that the $r = 0$ case is already done in \cite{ref:AFL}.
\todo{comment some basis thing}

\section{Parametrization of $\gamma$}
Again, assume $n = 3$.
Further assume $\gamma \in S_3(F)\rs$ is regular semisimple.
We identify some parameters for the orbit of $\gamma$
that we can use for our explicit calculations.

\subsection{Rewriting the orbital integral as a double integral over $E$
  via the group $H' \cong \GL_2(F)$}
Our orbital integral is at present a quadruple integral over $F$,
owing to $H = \GL_{2}(F)$ being a four-dimensional $F$-vector space.

It will be more economical to work with the orbital integral as a double integral
with two coefficients in $E$, in the following sense.
Define
\[ H' \coloneqq
  \left\{ \begin{bmatrix} t_1 & t_2 \\ \bar t_2 & \bar t_1 \end{bmatrix}
    \mid t_1, t_2 \in E \right\}
\]
which is indeed a four-dimensional $F$-algebra.
As before $H' \hookrightarrow G$ according to the same embedding
$\GL_2(E) \hookrightarrow \GL_(3)$
and so $H'$ also acts on $S_n(E)$ by conjugation.

As an $F$-algebra, we have an isomorphism (see \cite[\S4.1]{ref:AFL})
\begin{align*}
  \iota_2 \colon H = \GL_2(F)
  &\xrightarrow{\cong} H' \\
  \begin{bmatrix} a_{11} & a_{12} \\ a_{21} & a_{22} \end{bmatrix}
  &\mapsto \begin{bmatrix} t_1 & t_2 \\ \bar t_2 & \bar t_1 \end{bmatrix} \\
  t_1 &= \half\left( a_{11} + a_{22} + \frac{a_{12}}{\sqrt{\eps}} + a_{21} \sqrt{\eps} \right) \\
  t_2 &= \half\left( a_{11} - a_{22} + \frac{a_{12}}{\sqrt{\eps}} - a_{21} \sqrt{\eps} \right).
\end{align*}
Under this isomorphism, we have
\[ h \gamma h^{-1} = \iota_2(h) \gamma \overline{\iota_2(h)^{-1}}. \]
\todo{maybe i should actually check this to make sure I'm not crazy.}

This allows us to rewrite the orbital integral over $H'$ instead.
If we write $h' = \overline{\iota_2(h)^{-1}}$,
then the following integral formula is obtained.
\begin{proposition}
  [{\cite[\S4.2]{ref:AFL}}]
  \label{prop:orbital_over_H_prime}
  For brevity let $\eta(h') \coloneqq \eta(\det h')$ for $h' \in H'$.
  For $\gamma \in S_3(F)$, $\phi \in \HH(S_3(F), K')$, and $s \in \CC$,
  the orbital integral can instead be written as
  \[ \Orb(\gamma, \phi, s) =
    \int_{h' \in H'} \phi(\bar{h'}\inv \gamma h') \eta(h')
    \left\lvert \det(h') \right\rvert_F^{s} \odif{h'} \]
  where
  \[ \odif{h'} = \kappa \cdot \frac{\odif t_1 \odif t_2}
    {\left\lvert t_1 \bar t_1 - t_2 \bar t_2 \right\rvert_F^2} \]
  for the constant
  \[ \kappa \coloneqq \frac{1}{(1-q\inv)(1-q^{-2})}. \]
\end{proposition}

\subsection{Identifying a representative in the $H'$-orbit}
Evidently the orbital integral $\Orb(\gamma, \phi, s)$ in \Cref{prop:orbital_over_H_prime}
only depends on the $H'$-orbit of $\gamma$.
So it makes sense to pick a canonical representative for the $H'$-orbit to compute
the orbital integral in terms of.

Since we assumed $\gamma \in S_3(F)\rs$ is regular semisimple,
we can invoke \cite[Proposition 4.1]{ref:AFL}
to assume $\gamma$ is a representative specifically of the form
\[ \gamma(a,b,d) =
  \begin{bmatrix}
    a & 0 & 0 \\
    b & - \bar d & 1 \\
    c & 1 - d \bar d & d
  \end{bmatrix}
  \in S_3(F)\rs; \quad \text{where $c = -a \bar b + b d$} \]
over all $a \in E^1$, $b \in E$, $d \in E$ for which $(1-d\bar d)^2 - c \bar c \neq 0$.
In other words, the representatives described here cover all the regular orbits in $S_3(F)\rs$.

\subsection{Simplification due to the matching of non-quasi-split unitary group}
In this calculation, we restrict attention to the case where our regular $\gamma$
matches an element in the non-quasi-split unitary group.
\todo{I need to ask Wei exactly what's up here}
This is controlled by the parity of the invariant
\[ v\left( (1-d\bar d)^2 - c \bar c\right) \]
being odd.
Hence, we only have to consider this case:
\begin{assume}
  We will assume that
  \[ v\left( (1-d\bar d)^2 - c \bar c\right) \equiv 1 \pmod 2. \]
  \label{assume:u_odd}
\end{assume}

We will also see quite early on in our calculation
that the orbital integral vanishes if $v(d) < -r$.
Hence, we will always assume:
\begin{assume}
  $v(d) \geq -r$.
  \label{assume:vd_ge_minus_r}
\end{assume}

We will mostly be interested in the case where $v(b) = v(d) = 0$.
In fact, few other cases even occur at all given \Cref{assume:u_odd};
we will see momentarily that either $v(b) = v(d) \in \{-1, -2, \dots, -r\}$,
or one of $\{v(b), v(d)\}$ is zero and the other is nonnegative.

\section{Quantities to state the answer in terms of}
As we described earlier, our goal is to evaluate \eqref{eq:orbital_goal}
in terms of the parameters
\[ a \in E^1, \qquad b, d \in E, \qquad r \ge 0. \]
To simplify the notation in what follows,
it will be convenient to define several quantities that reappear frequently.
From \Cref{assume:u_odd}, we may define
\begin{equation}
  \delta \coloneqq v(1-d \bar d) = v(c) \neq -\infty.
  \label{eq:delta}
\end{equation}
Following \cite{ref:AFL} we will also define
\begin{equation}
  u \coloneqq \frac{\bar c}{1-d \bar d} \in \OO_E^\times
  \label{eq:u}
\end{equation}
so that $\nu(1-u \bar u) \equiv 1 \pmod 2$ and
\begin{equation}
  b = -au - \bar{d} \bar{u}.
  \label{eq:b}
\end{equation}
Note that this gives us the following repeatedly used identity
\begin{equation}
  b^2-4a\bar d = (au-\bar d \bar u)^2 - 4a\bar d(1-u\bar u).
  \label{eq:dos}
\end{equation}
Finally, define
\begin{equation}
  \ell \coloneqq v(b^2 - 4 a \ol d).
  \label{eq:ell}
\end{equation}
We will also define one additional parameter useful when $\ell$ is even
(but as we will see, redundant for odd $\ell$):
\begin{equation}
  \lambda \coloneqq v(1-u \bar u) \equiv 1 \pmod 2.
  \label{eq:lambda}
\end{equation}

Just as many pairs $(v(b), v(d))$ do not occur (given \Cref{assume:u_odd})
and $v(b) = v(d) = 0$ is the main case of interest,
the parameters $(\delta, \ell, \lambda)$ satisfy some additional relations.
We will now describe them.
\begin{proposition}
  \label{prop:parameter_constraints}
  Exactly one of the following situations is true:
  \begin{enumerate}[(a)]
    \ii $v(b) = v(d) = 0$, $\ell \ge 1$ is odd, $\ell < 2 \delta$, and $\lambda = \ell$.
    \ii $v(b) = v(d) = 0$, $\ell \ge 0$ is even, $\ell \le 2 \delta$, and $\lambda > \ell$ is odd.
    \ii $v(b) = 0$, $v(d) > 0$, $\ell = \delta = 0$, and ???
    \ii $v(b) > 0$, $v(d) = 0$, $\ell = 0$, $\delta \ge 0$, and ???
    \ii $v(b) = v(d) \in \{-1, \dots, -r\}$, $\ell = \delta = 2v(d) < 0$, and ???
  \end{enumerate}
  See \Cref{tab:parameter_constraints}.
  Moreover, whenever $\ell$ is even,
  the quantity $b^2 - 4 a \bar d$ is a square of some element in $E$.
\end{proposition}
\begin{proof}
  First assume $\ell$ is odd. We assert in this case we have
  \begin{equation}
    v(b) = v(d) = 0.
    \label{eq:odd_b_d_zero}
  \end{equation}
  Indeed if $v(d) \neq 0$, then $b = -au-\bar d\bar u$ is a unit,
  and hence so is $b^2 - 4 a \bar d$, causing $\ell = 0$, contradiction.
  And if $d$ is a unit, $\ell \neq 0$ means $v(b) = 0$ too.
  \todo{prove the result of (a)}

  For the rest of the proof we only consider even $\ell$.
  Because $b = -au - \bar d \bar u$, it cannot be the case that $v(b) > 0$ and $v(d) > 0$;
  moreover if either $v(b) < 0$ or $v(d) < 0$,
  then in fact $v(b) = v(d)$.
  \todo{wrap this up}

  Hence

  We now verify the last assertion that $b^2 - 4 a \bar d$
  is a square whenever $\ell$ is even.
  The proof in all cases uses \eqref{eq:dos} to show $b^2 - 4 a \bar d$
  is equal to $\varpi^\ell$ times a quadratic residue in $\OO_E^\times$.
  Indeed we need only verify that $v(4a \bar d(1 - u \bar u)) = v(d) + \lambda$
  has larger valuation than $v\left( (a u - \bar d \bar u)^2 \right) = \ell$.
  \begin{itemize}
    \ii In the case this follows from $\lambda > \ell$;
    \ii In the case where $v(d) > 0$ we have $\ell = 0$ and this is clear;
    \ii If $v(b) = v(d) < 0$ then $\ell = -2v(d)$ and this is clear too.
  \end{itemize}

\end{proof}

\begin{table}[ht]
  TODO: TABLE GOES HERE
  \caption{A table showing the five cases in \Cref{prop:parameter_constraints}.}
  \label{tab:parameter_constraints}
\end{table}
\todo{fill in the table}

In the case where $\ell$ is odd (and hence $\ell \ge 1$ and $v(b) = v(d) = 0$),
we get \eqref{eq:dos} implying $\lambda = \ell$
and thus $\lambda$ will never be used --- the orbital will be computed
as a function of $\ell$ and $\delta$ (and $r$).
However for even $\ell$ these numbers are never equal and our orbital
integral will be stated in terms of $\ell$, $\delta$, and $\lambda$ (and $r$).

\section{Result}
We can now state the answer.
\todo{put the differentiated result here}

\section{Setup for the orbital integral}
\subsection{Background}
\label{sec:orbital_background}

\subsubsection{Notation}
\begin{itemize}
  \ii Let $F$ be a finite extension of $\QQ_p$ for $p > 2$.
  \ii Let $E/F$ be an unramified quadratic field extension.
  \ii Denote by $\varpi$ a uniformizer of $\OO_F$, such that $\bar \varpi = \varpi$.
  \ii Let $q \coloneqq |F/\varpi|$ be the residue characteristic.
  \ii Let $v$ be the associated valuation for $\varpi$.
  \ii Let $\eta$ be the quadratic character attached to $E/F$ by class field theory,
  so that $\eta(x) = -1^{v(x)}$.
\end{itemize}

\subsubsection{Symmetric space}
We define the symmetric space
\[ S_3(F) \coloneqq \left\{ g \in \GL_3(E) \mid g \bar{g} = \id_3 \right\}. \]
It has a left action under $\GL_3(E)$ by $g \cdot s \mapsto gs\bar{g}\inv$.
\begin{lemma}
  [Cartan decomposition]
  For each integer $r \ge 0$ let
  \[ K'_{S,r} \coloneqq \GL_3(\OO_E) \cdot \begin{bmatrix} 0 & 0 & \varpi^r \\ 0 & 1 & 0 \\ \varpi^{-r} & 0 & 0 \end{bmatrix} \]
  denote the orbit of
  $\begin{bmatrix} 0 & 0 & \varpi^r \\ 0 & 1 & 0 \\ \varpi^{-r} & 0 & 0 \end{bmatrix}$
  under the left action of $\GL_3(\OO_E)$.
  Then we have a decomposition
  \[ S_3(F) = \coprod_{r \geq 0} K'_{S,r}. \]
\end{lemma}
The $r=0$ case will be given a special shorthand,
and can be expressed in a few equivalent ways:
\[ K'_S \coloneqq K'_{S,0}
  = \GL_3(\OO_E) \cdot \begin{bmatrix} & & 1 \\ & 1 \\ 1 \end{bmatrix}
  = \GL_3(\OO_E) \cdot \id_3 = S_3(F) \cap \GL_3(\OO_E). \]
One can equivalently define $K'_{S,r}$ to be the part of $S_3(F)$
for which the most negative valuation among the nine entries is $-r$.

For $r \geq 0$, define
\[ K'_{S, \le r} \coloneqq S_3(F) \cap \varpi^{-r} \GL_3(\OO_E). \]
We can re-parametrize the problem according to the following claim.
\begin{claim}
  \[ K'_{S, \le r} = K'_{S,0} \sqcup K'_{S,1} \sqcup \dots \sqcup K'_{S,r}. \]
\end{claim}
If this claim is true (still need to check it),
then an integral over each $K'_{S, \le r}$ lets us extract the integrals over $K'_{S,r}$.

\subsubsection{Orbital integral}
Define
\[ H' \coloneqq
  \left\{ \begin{bmatrix} t_1 & t_2 \\ \bar t_2 & \bar t_1 \end{bmatrix} \right\}
  \cong \GL_2(F). \]
We embed $H'$ into $\GL_3(F)$ by
$h' \mapsto \left[ \begin{smallmatrix} h' & 0 \\ 0 & 1 \end{smallmatrix} \right]$,
which allows $H$ to act on $\GL_3(F)$ and hence $S_3(F)$.

Now we can define the orbital integral.
\begin{definition}
  For brevity let $\eta(h') \coloneqq \eta(\det h')$ for $h' \in H'$.
  For $\gamma \in S_3(F)$ and $s \in \CC$, we define the orbital integral by
  \[ \Orb(\gamma, \phi, s) \coloneqq
    \int_{g \in H'} \phi(\bar g\inv \gamma g) \eta(g)
    \left\lvert \det(g) \right\rvert_F^{-s} \odif g \]
  where
  \[ \odif g = \kappa \cdot \frac{\odif t_1 \odif t_2}
    {\left\lvert t_1 \bar t_1 - t_2 \bar t_2 \right\rvert_F^2} \]
  for the constant
  \[ \kappa \coloneqq \frac{1}{(1-q\inv)(1-q^{-2})}. \]
\end{definition}

Indeed, for $h' \in H$ and $\gamma \in S_3(F)$ we have $h' \gamma (\bar h')\inv \in S_3(F)$
and so the indicator function is filtering based on which part of the
Cartan decomposition that $h' \gamma (\bar h')\inv$ falls in.

Evidently $\Orb(\gamma, \phi, s)$ only depends on the $H'$-orbit of $\gamma$.
So it makes sense to pick a canonical representative for the $H'$-orbit to compute
the orbital integral in terms of.
For so-called \emph{regular} $\gamma$, the representatives
\[ \gamma(a,b,d) =
  \begin{bmatrix}
    a & 0 & 0 \\
    b & - \bar d & 1 \\
    c & 1 - d \bar d & d
  \end{bmatrix}
  \in S_3(F); \quad \text{where $c = -a \bar b + b d$} \]
over all $a \in E^1$, $b \in E$, $d \in E$ for which $(1-d\bar d)^2 - c \bar c \neq 0$,
cover all the \emph{regular} orbits, which are the ones we care about.

The paper \cite{ref:AFL} computes $\pdv{}{s}\Orb(\gamma, \mathbf{1}_{K'_S}, s)$ at $s=0$
in terms of $a$, $b$, $d$.
Our goal is to compute for
\[ \pdv{}{s}\Orb(\gamma, \mathbf{1}_{K'_{S, \le r}}, s) \]
at $s=0$ for any $r > 0$ as well.

\subsection{Reparametrization in terms of valuations}
\subsubsection{Computation of value in indicator function}
We are integrating over $t_1 \in E$ and $t_2 \in E$.
Regarding $g \in H'$ as an element of $\GL_3$ as described before, we have
\[ g = \begin{bmatrix}
  t_1 & t_2 & 0  \\
  \bar t_2 & \bar t_1 & 0 \\
  0 & 0 & 1
  \end{bmatrix}. \]
We therefore have
\[ \bar g \inv = \begin{bmatrix}
  \frac{t_1}{t_1 \bar t_1 - t_2 \bar t_2} & \frac{-\bar t_2}{t_1 \bar t_1 - t_2 \bar t_2} & 0 \\
  \frac{-t_2}{t_1 \bar t_1 - t_2 \bar t_2} & \frac{\bar t_1}{t_1 \bar t_1 - t_2 \bar t_2} & 0 \\
  0 & 0 & 1 \end{bmatrix}. \]
Hence
\begin{align*}
  \bar g \inv \gamma g
  &=
  \begin{bmatrix}
  \frac{t_1}{t_1 \bar t_1 - t_2 \bar t_2} & \frac{-\bar t_2}{t_1 \bar t_1 - t_2 \bar t_2} & 0 \\
  \frac{-t_2}{t_1 \bar t_1 - t_2 \bar t_2} & \frac{\bar t_1}{t_1 \bar t_1 - t_2 \bar t_2} & 0 \\
  0 & 0 & 1 \end{bmatrix}
  \begin{bmatrix}
    a & 0 & 0 \\
    b & - \bar d & 1 \\
    c & 1 - d \bar d & d
  \end{bmatrix}
  \begin{bmatrix}
  t_1 & t_2 & 0  \\
  \bar t_2 & \bar t_1 & 0 \\
  0 & 0 & 1
  \end{bmatrix} \\
  &=
  \begin{bmatrix}
  \frac{t_1}{t_1 \bar t_1 - t_2 \bar t_2} & \frac{-\bar t_2}{t_1 \bar t_1 - t_2 \bar t_2} & 0 \\
  \frac{-t_2}{t_1 \bar t_1 - t_2 \bar t_2} & \frac{\bar t_1}{t_1 \bar t_1 - t_2 \bar t_2} & 0 \\
  0 & 0 & 1 \end{bmatrix}
  \begin{bmatrix}
    at_1 & at_2 & 0 \\
    bt_1 - \bar d \bar t_2 & b t_2 - \bar d \bar t_1 & 1 \\
    ct_1 + (1-d\bar d)\bar t_2 & ct_2 + (1-d \bar d) \bar t_1 & d
  \end{bmatrix}
  \\
  &=
  \begin{bmatrix}
    \dfrac{at_1^2 - bt_1 \bar t_2 + d \bar t_2^2}{t_1 \bar t_1 - t_2 \bar t_2}
    & \dfrac{at_1t_2 - bt_2 \bar t_2 + \bar d \bar t_1 \bar t_2}{t_1 \bar t_1 - t_2 \bar t_2}
    & \dfrac{-\bar t_2}{t_1 \bar t_1 - t_2 \bar t_2} \\[2ex]
    \dfrac{-at_1t_2+bt_1\bar t_1-\bar d \bar t_1 \bar t_2}{t_1 \bar t_1 - t_2 \bar t_2}
    & \dfrac{-at_2^2+b\bar t_1 t_2-d\bar t_1^2}{t_1 \bar t_1 - t_2 \bar t_2}
    & \dfrac{\bar t_1}{t_1 \bar t_1 - t_2 \bar t_2} \\[2ex]
    ct_1 + (1-d\bar d)\bar t_2 & ct_2 + (1-d \bar d) \bar t_1 & d
  \end{bmatrix}
\end{align*}
Let us define \[ t = t_2 \bar t_1 \inv \iff t_2 = t \bar t_1. \]
This lets us rewrite everything in terms of the ratio $t$ and $t_1 \in E$:
\[
  \bar g \inv \gamma g
  =
  \begin{bmatrix}
    \dfrac{t_1^2(a-b\bar t+\bar d \bar t^2)}{t_1 \bar t_1(1-t \bar t)}
    & \dfrac{t_1 \bar t_1(at-bt\bar t+\bar d \bar t)}{t_1 \bar t_1(1-t \bar t)}
    & \dfrac{t_1 \cdot (-\bar t)}{t_1 \bar t_1 (1-t \bar t)} \\[2ex]
    \dfrac{t_1\bar t_1(-at+b-\bar d \bar t)}{t_1 \bar t_1(1-t \bar t)}
    & \dfrac{\bar t_1^2(-at^2+bt-\bar d)}{t_1 \bar t_1(1-t \bar t)}
    & \dfrac{-\bar t_1}{t_1 \bar t_1(1-t \bar t)} \\[2ex]
    t_1(c + (1-d\bar d)\bar t) & \bar t_1(ct + (1-d \bar d)) & d
  \end{bmatrix}
\]
This new parametrization is better because $t_1$ only plays the role of
a scale factor on the outside, with ``interesting'' terms only involving $t$.
To make this further explicit, we write
\[ t_1 = \varpi^{-m} \epsilon \]
for $m \in \ZZ$ and $\epsilon \in \OO_E^\times$.
Then we actually have
\[
  \begin{bmatrix} \bar\epsilon \\ & \epsilon \\ & & 1 \end{bmatrix}
  \bar g \inv \gamma g
  \begin{bmatrix} \epsilon\inv \\ & \bar\epsilon\inv \\ & & 1 \end{bmatrix}
  =
  \begin{bmatrix}
    \dfrac{a-b\bar t+\bar d \bar t^2}{1-t \bar t}
    & \dfrac{at-bt\bar t+\bar d \bar t}{1-t \bar t}
    & \dfrac{-\varpi^m \bar t}{1-t \bar t} \\[2ex]
    \dfrac{-at+b-\bar d \bar t}{1-t \bar t}
    & \dfrac{-at^2+bt-\bar d}{1-t \bar t}
    & \dfrac{-\varpi^m}{1-t \bar t} \\[2ex]
    \dfrac{c + (1-d\bar d)\bar t}{\varpi^m} & \dfrac{ct + (1-d \bar d)}{\varpi^m} & d
  \end{bmatrix}
\]
For brevity, we will let $\Gamma(\gamma, t, m)$ denote the right-hand matrix.
The conjugation by
$\left[ \begin{smallmatrix} \epsilon\inv \\ & \bar\epsilon\inv \\ & & 1 \end{smallmatrix} \right]$
has no effect on any of the $K'_{S, \le r}$, so that we can simply use
\[ \mathbf{1}_{K'_{S, \le r}}(\bar g \inv \gamma g) = \mathbf{1}_{K'_{S, \le r}}(\Gamma(\gamma, t, m)) \]
in the work that follows.
For brevity, we abbreviate
\[ \mathbf{1}_{\le r}(\gamma, t, m) \coloneqq \mathbf{1}_{K'_{S, \le r}}(\Gamma(\gamma, t, m)). \]

\subsubsection{Reparametrizing the integral in terms of $t$ and $m$}
From now on, following \cite{ref:AFL} we always fix the notation
\begin{align*}
  m &= m(t_1) \coloneqq -v(t_1) \\
  n &= n(t) \coloneqq v(1-t\bar t).
\end{align*}
We need to rewrite the integral, phrased originally via $\odif g$,
in terms of the parameters $t$ (hence $n$), $m$, and $\gamma$.
We start by observing that
\[ \det g = t_1 \bar t_1 - t_2 \bar t_2 = t_1 \bar t_1 (1 - t\bar t) \]
which means that
\[ v(\det g) = -2m + n \]
ergo
\begin{align*}
  \left\lvert \det g \right\rvert_F &= q^{-v(\det g)} = q^{2m-n} \\
  \eta(g) &= (-1)^{v(\det g)} = (-1)^n.
\end{align*}
Meanwhile, from $t_2 = t \bar t_1$ we derive
\[ \odif t_2 = \left\lvert t_1 \right\rvert_E \odif t = q^{2m} \odif t. \]

Bringing this all into the orbital integral gives
\begin{align*}
  \Orb(\gamma, \mathbf{1}_{K'_{S, \le r}} s)
  &= \kappa \int_{t, t_1 \in E} \mathbf{1}_{\le r}(\gamma, t, m)
  (-1)^n \left( q^{2m-n} \right)^{s-2} \odif t_1 \cdot (q^{2m} \odif t) \\
  &= \kappa \int_{t, t_1 \in E} \mathbf{1}_{\le r}(\gamma, t, m)
  (-1)^n q^{s(2m-n)} \cdot q^{2n-2m} \odif t \odif t_1.
\end{align*}

\subsection{Setup}
\subsubsection{Simplifying assumptions}
For the purposes of \cite{ref:AFL},
we will only care about the following case:
\begin{assume}
  \[ v\left( (1-d\bar d)^2 - c \bar c\right) \equiv 1 \pmod 2 \]
  \label{assume:u_odd}
\end{assume}
\todo{I need to ask Wei exactly why we're only doing this case}

We will thus also assume:
\begin{assume}
  $v(d) \geq -r$.
\end{assume}
This is fine because if this $v(d) < -r$ then the integral will always vanish
(because the bottom-right entry of $\Gamma(\gamma, t, m)$ is no-good).

We will really mostly be interested in the case where $v(b) = v(d) = 0$.
In fact, few other cases even occur at all
given \Cref{assume:u_odd};
we will see momentarily that either
$v(b) = v(d) \in \{-1, -2, \dots, -r\}$,
or one of $\{v(b), v(d)\}$ is zero and the other is nonnegative.

\subsubsection{Notations}
As we described earlier, our goal is to give an answer in terms of
\[ a \in E^1, \qquad b, d \in E, \qquad r \ge 0. \]
To simplify the notation in what follows,
it will be convenient to define several quantities that reappear frequently.
From \Cref{assume:u_odd}, we may define
\begin{equation}
  \delta \coloneqq v(1-d \bar d) = v(c) \neq -\infty.
  \label{eq:delta}
\end{equation}
Following \cite{ref:AFL} we will also define
\begin{equation}
  u \coloneqq \frac{\bar c}{1-d \bar d} \in \OO_E^\times
  \label{eq:u}
\end{equation}
so that $\nu(1-u \bar u) \equiv 1 \pmod 2$ and
\begin{equation}
  b = -au - \bar{d} \bar{u}.
  \label{eq:b}
\end{equation}
Note that this gives us the following repeatedly used identity
\begin{equation}
  b^2-4a\bar d = (au-\bar d \bar u)^2 - 4a\bar d(1-u\bar u).
  \label{eq:dos}
\end{equation}
Finally, define
\begin{equation}
  \ell \coloneqq v(b^2 - 4 a \ol d).
  \label{eq:ell}
\end{equation}
We will also define one additional parameter useful when $\ell$ is even
(but as we will see, redundant for odd $\ell$):
\begin{equation}
  \lambda \coloneqq v(1-u \bar u) \equiv 1 \pmod 2.
  \label{eq:lambda}
\end{equation}

Just as many pairs $(v(b), v(d))$ do not occur (given \Cref{assume:u_odd})
and $v(b) = v(d) = 0$ is the main case of interest,
the parameters $(\delta, \ell, \lambda)$ satisfy some additional relations.
We will now describe them.
\begin{proposition}
  Exactly one of the following situations is true.
  \begin{enumerate}[a.]
    \ii $v(b) = v(d) = 0$, $\ell \ge 1$ is odd,
      $\ell < 2 \delta$, and $\lambda = \ell$.
    \ii $v(b) = v(d) = 0$, $\ell \ge 0$ is even,
      $\ell \le 2 \delta$, and $\lambda > \ell$ is odd.
    \ii $v(b) = 0$, $v(d) > 0$, $\ell = \delta = 0$, and ???
    \ii $v(b) > 0$, $v(d) = 0$, $\ell  = 0$, $\delta \ge 0$, and ???
    \ii $v(b) = v(d) \in \{-1, \dots, -r\}$,
    $\ell = \delta = 2v(d) < 0$, and ???
  \end{enumerate}
  \label{prop:parameter_constraints}
\end{proposition}
\begin{proof}
  \todo{Write this out; import from previous if needed.
    Also figure out exactly what the constraints on $\lambda$ are.}
\end{proof}

\todo{imported this; need to rearrange}
\begin{proposition}
  Whenever $\ell$ is odd, we must have
  \begin{equation}
    v(b) = v(d) = 0.
    \label{eq:odd_b_d_zero}
  \end{equation}
\end{proposition}
\begin{proof}
  [Proof of \eqref{eq:odd_b_d_zero}]
  If $v(d) \neq 0$, then $b = -au-\bar d\bar u$ is a unit,
  and hence so is $b^2 - 4 a \bar d$, causing $\ell = 0$, contradiction.
  And if $d$ is a unit, $\ell \neq 0$ means $v(b) = 0$ too.
\end{proof}

In the case where $\ell$ is odd (and hence $\ell \ge 1$ and $v(b) = v(d) = 0$),
we get \eqref{eq:dos} implying $\lambda = \ell$
and this definition will never be used --- the orbital will be computed
as a function of $\ell$ and $\delta$ (and $r$).
However for even $\ell$ these numbers are never equal and our orbital
integral will be stated in terms of $\ell$, $\delta$, and $\lambda$ (and $r$).

\subsection{Description of the support of $\mathbf{1}_{\le r}$ when $n \le 0$}
\begin{claim}
  Whenever $n = 0$ (this requires $v(t) \geq 0$),
  \[
    \mathbf{1}_{\le r}(\gamma, t, m) =
    \begin{cases}
      1 & \text{if } -r \le m \le \delta+r \\
      0 & \text{otherwise.}
    \end{cases}
  \]
\end{claim}
\begin{proof}
  We have to consider the nine entries of $\Gamma(\gamma, t, m)$ in tandem.

  The upper $2 \times 2$ matrix is always in $\omega^{-r}\OO_E$,
  because $v(t) \geq 0$, $v(d) \geq -r$, $v(b) \geq -r$, and $v(a) = 0$ suffices.

  In the right column, since $v(t) \geq 0$ and $n = 0$, the condition is simply $m \ge -r$.

  In the bottom row, we need
  $v\left( c+(1-d\bar d) \bar t \right)-m \geq -r$
  and $v\left( ct +(1-d\bar d) \right)-m \geq -r$.
  If $v(t) > 0$ this is equivalent to $m-r \leq \delta$.
  In the case where $v(t) = 0$ we instead use the observation that
  \begin{equation}
    \left[ c + (1-d \bar d) \bar t \right]
    - \bar t \left[ ct + (1-d \bar d) \right] = (1-t\bar t) c
    \label{eq:ctrick}
  \end{equation}
  which forces at least one of $ct + (1-d \bar d)$ and $c + (1-d \bar d) \bar t$ to
  have valuation $\delta$. So the claim follows now.
\end{proof}

\begin{claim}
  Suppose $n = -2k < 0$, equivalently, $v(t) = -k < 0$, for some $k$.
  \[
    \mathbf{1}_{\le r}(\gamma, t, m) =
    \begin{cases}
      1 & \text{if } -r \le m+k \le \delta+r \\
      0 & \text{otherwise.}
    \end{cases}
  \]
\end{claim}
\begin{proof}
  The proof is similar to the previous claim, but simpler.

  Since $k > 0$, the fraction $\frac{t^2}{1-t \bar t}$ has positive valuation,
  so the upper $2 \times 2$ of $\Gamma(\gamma, t, m)$ is always in $\varpi^{-r}\OO_E$.
  Turning to the right column, the condition reads exactly $m+k \geq -r$.
  Finally, in the bottom row, from $v(t) > 0$ and $v(c) = \delta$
  the condition is simply $-k+\delta-m \geq -r$.
\end{proof}

\subsection{Description of the support of $\mathbf{1}_{\le r}$ when $n > 0$}
In this situation we evaluate over $n > 0$ only.
In this case $t$ is automatically a unit.

\subsubsection{Volume lemma}
The following two lemmas will be useful.

\begin{lemma}
  Let $\xi \in \OO_E^\times$, $\rho \in \ZZ$, and $n \ge \max(\rho, 1)$ an integer.
  Then
  \begin{align*}
    &\Vol\left( \left\{ x \in E \mid v(1-x \bar x) = n,
      \; v(x-\xi) \ge \rho \right\} \right) \\
    &=
    \begin{cases}
      0 & \text{if } v(1-\xi\bar\xi) < \rho \\
      q^{-n}(1-q^{-2}) & \text{if } \rho \le 0 \\
      q^{-(n+\rho)}(1-q\inv) & \text{if } v(1-\xi\bar\xi) \ge \rho \ge 1.
    \end{cases}
  \end{align*}
  \label{lem:volume}
\end{lemma}
\begin{proof}
  The case $\rho > 0$ is proved in \cite[Lemma 4.4]{ref:AFL}.

  When $\rho \le 0$, the condition $v(x - \xi) \ge \rho$ is vacuously true,
  so we just are computing
  $\Vol\left( \left\{ x \in E \mid v(1-x \bar x) = n \right\} \right)$.
  Follows from summing the previous lemma over $\rho = 1$
  and $\xi \in \OO_E / \varpi$ (there are $q_E-1 = q^2-1$ choices for $\xi$).
  \todo{This doesn't actually check out. Check margin of page 235 of notebook.}
\end{proof}

We also comment on the well-known fact that in an ultrametric space,
any two disks are either disjoint or one is contained in the other.
(In other words, the Mastercard logo cannot be drawn. See \Cref{fig:no_mastercard}.)
\begin{proposition}
  Choose $\xi_1, \xi_2 \in E$ and $\rho_1 \geq \rho_2$.
  Consider the two disks:
  \begin{align*}
    D_1 &= \left\{ x \in E \mid v(x-\xi_1) \ge \rho_1 \right\} \\
    D_2 &= \left\{ x \in E \mid v(x-\xi_2) \ge \rho_2 \right\}.
  \end{align*}
  Then, if $v(\xi_1-\xi_2) \geq \rho_2$, we have $D_1 \subseteq D_2$.
  If not, instead $D_1 \cap D_2 = \varnothing$.
  \label{prop:no_mastercard}
\end{proposition}
\begin{proof}
  Because $E$ is an ultrametric space and $\Vol(D_1) \leq \Vol(D_2)$,
  we either have $D_1 \subseteq D_2$ or $D_1 \cap D_2 = \varnothing$.
  The latter condition checks which case we are in by testing if $\xi_1 \in D_2$,
  since $\xi_1 \in D_1$.
\end{proof}
\todo{is it $q$ or $q_E$}

\begin{figure}
\begin{asy}
  defaultpen(fontsize(12pt));
  size(7cm);
  pair O1 = 0.57*dir(-30);
  pair O2 = (0,0);
  real r1 = 0.4;
  real r2 = 1;
  filldraw(circle(O2, r2), rgb(0.9, 0.9, 0.9));
  filldraw(circle(O1, r1), rgb(0.8, 0.8, 0.9));
  pair P1 = O1+r1*dir(230);
  pair P2 = O2+r2*dir( 70);
  draw(O1--P1);
  draw(O2--P2);
  dot("$\xi_1$", O1, dir(90));
  dot("$\xi_2$", O2, dir(180));
  label("$q^{-\rho_1}$", midpoint(O1--P1), dir(P1-O1)*dir(90));
  label("$q^{-\rho_2}$", midpoint(O2--P2), dir(P2-O2)*dir(90));
\end{asy}
\caption{Figure corresponding to \Cref{prop:no_mastercard}.}
\label{fig:no_mastercard}
\end{figure}

We package both of these results together
lemma that will be used repeatedly.
\begin{lemma}
  \label{lem:quadruple_ineq}
  Let $\xi_1, \xi_2 \in \OO_E^\times$ and let $\rho_1 \ge \rho_2$ be integers.
  Also let $n \ge \max(\rho_1, 1)$ be an integer.
  Then the set of points $x \in E$ satisfying all of the equations
  \begin{align*}
    v(x - \xi_1) &\ge \rho_1 \\
    v(x - \xi_2) &\ge \rho_2 \\
    v(1 - x \ol x) &= n
  \end{align*}
  has positive volume if and only if
  \[ v(1 - \xi_1 \bar{\xi_1}) \ge \rho_1, \qquad \rho_2 \le v(\xi_1 - \xi_2). \]
  In that case, the volume is equal to
  \[
    \begin{cases}
      q^{-(n+\rho_1)}(1-q\inv) & \text{if } \rho_1 \ge 1 \\
      q^{-n}(1-q^{-2}) & \text{if } \rho_1 \le 0.
    \end{cases}
  \]
\end{lemma}
In the situation where $\xi_i \notin \OO_E^\times$,
the condition $v(x-\xi_i) = \min(0, v(\xi_i))$ becomes independent of the value of $x$,
and so \Cref{lem:quadruple_ineq} becomes unnecessary
(\Cref{lem:volume} will suffice).
We will deal with this situation when it arises.

\subsubsection{Setup}
Consider the upper $2 \times 2$ matrix of $\Gamma(\gamma, t, m)$.
Using the identities
\begin{align*}
  \dfrac{a-b\bar t+\bar d \bar t^2}{1-t \bar t}
    - \bar t \cdot \dfrac{at-bt\bar t+\bar d \bar t}{1-t \bar t}
    &= a-b\bar t \in \varpi^{-r} \OO_E \\[2ex]
  \dfrac{a-b\bar t+\bar d \bar t^2}{1-t \bar t}
    + \bar t \cdot \dfrac{-at+b-\bar d \bar t}{1-t \bar t}
    &= a \in \varpi^{-r} \OO_E \\[2ex]
  \dfrac{-at+b-\bar d \bar t}{1-t \bar t}
    - \bar t \cdot \dfrac{-at^2+bt-\bar d}{1-t \bar t}
    &= -a+b \in \varpi^{-r} \OO_E,
\end{align*}
it follows that as soon as one entry is in $\varpi^{-r} \OO_E$, they all are.
Meanwhile, the requirements on the other entries amount to
\begin{align}
  m & \geq n - r \\
  v\left( c+(1-d \bar d) \bar t \right) &\geq m-r \label{eq:cddtop} \\
  v\left( ct+(1-d \bar d) \right) &\geq m-r \label{eq:cddbot}
\end{align}
According to the earlier identity \eqref{eq:ctrick},
if \eqref{eq:cddtop} is assumed true,
then \eqref{eq:cddbot} is equivalent to
\[ \delta + v(1-t \bar t) \ge m-r. \]
Meanwhile, since $v(c+(1-d \bar d) \bar t) = v(\bar c + (1-d \bar d)t)$,
\eqref{eq:cddtop} is itself equivalent to
\[ v(t+u) + \delta \geq m-r \]
by reading the definition of \eqref{eq:u}.

Finally, we use a tricky substitution
\[ (2at-b)^2 - (b^2-4a\bar d) = -4a(-at^2+bt-\bar d) \]
to rewrite $v(-at^2+bt-\bar d) \geq n-r$
as $v\left( (2at-b)^2 - (b^2-4a\bar d) \right) \geq n-r$.

In summary:
\begin{claim}
  Assume $t$ is such that $n = v(1-t \bar t) > 0$.
  Then $\mathbf{1}_{\le r}(\gamma, t, m) = 1$ if and only if
  \[ n - r \leq m \leq n + \delta + r \]
  and $t$ lies in the set specified by
  \begin{align*}
    v\left( (2at-b)^2 - (b^2-4a\bar d) \right) &\geq n-r \\
    v(t+u) &\ge m-\delta-r.
  \end{align*}
\end{claim}

\subsubsection{Rewriting the quadratic constraint on the valuation of $t$}
We now analyze the inequality
\begin{equation}
  v\left( (2at-b)^2 - (b^2-4a\bar d) \right) \geq n-r
  \label{eq:2atb_dos}
\end{equation}
and divide it into several (disjoint) possibilities.
Recalling that $\ell = b^2 - 4 a \bar d$, there are three possibilities:
\begin{itemize}
  \ii If $\ell \ge n-r$, then \eqref{eq:2atb_dos} is equvialent to
  \[ 2v(2at-b) \ge n-r
    \iff v\left( t - \frac{b}{2a} \right) \ge \left\lceil \frac{n-r}{2} \right\rceil. \]
  We will further subdivide this into two cases.
  \begin{itemize}
    \ii \textbf{Case 1} is the situation where $\left\lceil \frac{n-r}{2} \right\rceil \ge m - \delta - r$.
    \ii \textbf{Case 2} is the situation where $\left\lceil \frac{n-r}{2} \right\rceil < m - \delta - r$.
  \end{itemize}

  \ii If $\ell < n-r$, then \eqref{eq:2atb_dos} could only hold if $v(2at-b) = \frac{\ell}{2}$.
  Note that in particular, this requires $\ell$ to be even.
  If this happens, then $b^2 - 4 a \bar d$ must be a square and we denote it $\tau^2$.
  Thus, \eqref{eq:2atb_dos} then reads
  \[ v(2at-b+\tau) + v(2at-b-\tau) \ge n-r. \]
  Since we are assuming $n > \ell + r$,
  it must be the case that one of the two factors
  $v(2at-b\mp\tau)$ is equal to $v(\tau) = \ell / 2$ exactly.

  Suppose $v(2at-b+\tau) = \frac{\ell}{2}$, so we need
  \[ v\left( t - \frac{b+\tau}{2a} \right) = v(2at-b-\tau) \ge n - \frac{\ell}{2} - r. \]
  We further subdivide this into two cases:
  \begin{itemize}
    \ii \textbf{Case 3\ts+} is the situation where $n - \frac{\ell}{2} - r > m - \delta - r$.
    \ii \textbf{Case 4\ts+} is the situation where $n - \frac{\ell}{2} - r \le m - \delta - r$.
  \end{itemize}
  Replacing $\tau$ with $-\tau$ above gives us two additional cases
  which we denote \textbf{Case 3\ts-} and \textbf{Case 4\ts-}.
\end{itemize}

This gives us six cases, with each $t \in E$ satisfying at most one of them.
(If $\ell$ is odd, only \textbf{Case 1} and \textbf{Case 2} are used.)
In each case, for a given pair $(n,m)$ we are interested in the volume of $t$
such that two disk inequalities hold together with the assumption $n = v(1-t \bar t)$.

We rewrite these six cases in the format specified by \Cref{lem:quadruple_ineq},
noting that each possibility will actually split into two sub-cases
(although the lemma will only apply in the cases where the centers
$\xi_i$ are actually in $\OO_E^\times$;
this which will be true in the main case $v(b) = v(d) = 0$).
This gives the \Cref{tab:orbital_cases} below;

\begin{table}[ht]
  \centering
  \begin{tabular}{ll cccc cc}
    & Assume & $v(1-\xi_1 \bar{\xi_1})$ & $\rho_1$ and $\rho_2$ & $v(\xi_1-\xi_2)$ & $\xi_1$ & $\xi_2$ \\ \hline
    \textbf{1} & $n \le \ell+r$
        & $v(4-b\bar{b})$
        & $\left\lceil \frac{n-r}{2} \right\rceil \ge m-\delta-r$
        & $v(au - \bar d \bar u)$
        & $\frac{b}{2a}$ & $-u$ \\
    \textbf{2} & $n \le \ell+r$
        & $\lambda$
        & $m-\delta-r > \left\lceil \frac{n-r}{2} \right\rceil$
        & $v(au- \bar d \bar u)$
        & $-u$ & $\frac{b}{2a}$ \\ \hline
    \textbf{3\ts+} & $n > \ell + r$
          & $v(1-\frac{\Norm(b+\tau)}{4})$
          & $n-\frac{\ell}{2}-r > m-\delta-r$
          & $v(au-\bar d \bar u+\tau)$
          & $\frac{b+\tau}{2a}$ & $-u$ \\
    \textbf{3\ts-} & $n > \ell + r$
          & $v(1-\frac{\Norm(b-\tau)}{4})$
          & $n-\frac{\ell}{2}-r > m-\delta-r$
          & $v(au-\bar d \bar u-\tau)$
          & $\frac{b-\tau}{2a}$ & $-u$ \\
    \textbf{4\ts+} & $n > \ell + r$
          & $v(1-\frac{\Norm(b+\tau)}{4})$
          & $m-\delta-r \ge n-\frac{\ell}{2}-r$
          & $v(au-\bar d \bar u+\tau)$
          & $-u$ & $\frac{b+\tau}{2a}$ \\
    \textbf{4\ts-} & $n > \ell + r$
          & $v(1-\frac{\Norm(b+\tau)}{4})$
          & $m-\delta-r \ge n-\frac{\ell}{2}-r$
          & $v(au-\bar d \bar u-\tau)$
          & $-u$ & $\frac{b-\tau}{2a}$ \\
  \end{tabular}
  \caption{The six cases for calculating the orbital integral.}
  \label{tab:orbital_cases}
\end{table}
Note that in generating this table, we did the calculations
\begin{align*}
  v\left( u + \frac{b}{2a} \right)
  &= v\left( \frac{au-\bar d \bar u}{2a} \right)
  = v(au - \bar d \bar u) \\
  v\left( u + \frac{b \pm \tau}{2} \right)
  &= v(au - \bar d \bar u \pm \tau).
\end{align*}
to populate the entries for $v(\xi_1 - \xi_2)$,
as well as the identity
\[ 1 - \frac{b \pm \tau}{2a} \cdot \frac{\bar b \pm \bar \tau}{2\bar a}
= \frac{4 - \Norm(b \pm \tau)}{4} \]
to calculate $v(1-\xi_1\bar{\xi_1})$ entries in the latter four cases.

\subsubsection{Analysis of Case 1 and 2 assuming $n > 0$ and $v(b) = v(d) = 0$.}
We analyze Case 1 and 2 assuming $v(b) = v(d) = 0$.

Considering $n > 0$ and $n-r \le m \le n+\delta+r$ as fixed,
we compute the volume of the set of $t$
for which $n = v(1-t\bar t)$ and $\mathbf{1}_{\le r}(\gamma,t,m) = 1$.

In addition to the constraint $n \le \ell + r$,
we see that the two cases have the following additional requirements:
\begin{description}
  \ii[Case 1.] If $m < \left\lceil \frac{n-r}{2} \right\rceil + \delta + r$ then we need
  \begin{align}
    v(4-b\bar b) &\ge \left\lceil \frac{n-r}{2} \right\rceil \label{eq:odd_ineq1} \\
    v(au - \bar d \bar u) &\ge m - \delta - r \label{eq:odd_ineq2}.
  \end{align}

  \ii[Case 2.] If $m \geq \left\lceil \frac{n-r}{2} \right\rceil + \delta + r$ then we need
  \begin{align}
    \lambda = v(1-u \bar u) &\ge m-\delta-r \label{eq:odd_ineq3} \\
    v(au - \bar d \bar u) &\ge \left\lceil \frac{n-r}{2} \right\rceil \label{eq:odd_ineq4}.
  \end{align}
\end{description}

We will now show that some of these inequalities are redundant and can be ignored.

\begin{claim}
  \eqref{eq:odd_ineq2} and \eqref{eq:odd_ineq4} are redundant
  i.e.\ they are automatically true for $0 < n \le \ell + r$.
\end{claim}
\begin{proof}
  First, assume that $\ell$ is odd.
  Then \eqref{eq:dos} together with $v(b) = v(d) = 0$ gives
  \begin{equation}
    \lambda = \ell = v(1-u \bar u) < 2v(au - \bar d \bar u).
    \label{eq:odd_center_distance}
  \end{equation}
  On the other hand, if $\ell$ is even, then we have instead
  \begin{equation}
    \ell = 2v(au - \bar d \bar u) < v(1-u\bar u) = \lambda
    \label{eq:even_center_distance_case12}
  \end{equation}
  Thus, regardless of the parity of $\ell$, we always have
  \[ v(au - \bar d \bar u) \ge \frac{\ell}{2} \ge \frac{n-r}{2}. \qedhere \]
\end{proof}

\begin{claim}
  \eqref{eq:odd_ineq1} is redundant.
\end{claim}
\begin{proof}
  The equation
  \[ (4-b \bar b) = -4au(1-d\bar d) - \bar b(b^2-4a\bar d) \]
  implies
  \begin{equation}
    v(4-b\bar b) \ge \min(\ell,\delta) \text{ with equality if } \ell \neq \delta.
    \label{eq:bb_odd}
  \end{equation}
  Hence, a priori \eqref{eq:bb_odd} suggests that we have a condition
  $n \le r + 2 \delta$ in addition to $n \le r + \ell$.
  However, by \Cref{prop:parameter_constraints}, we always have $\ell \le 2 \delta$,
  and consequently \eqref{eq:odd_ineq1} is redundant as well.
\end{proof}

Putting all of this together, we find that the valid pairs $(n,m)$ come in two cases.

\paragraph{Double sum for Case 1.}
We sum over $(m,n)$ such that
\begin{equation}
  \begin{aligned}
    1 &\leq n \leq \ell + r, \\
    n-r &\leq m \leq \left\lceil \frac{n-r}{2} \right\rceil+\delta+r - 1
  \end{aligned}
  \label{eq:odd_range1}
\end{equation}
where each $(m,n)$ gives a volume contribution of
\[
  \begin{cases}
    q^{-n - \left\lceil \frac{n-r}{2} \right\rceil} \left( 1 - q\inv \right)
      & \text{if $n > r$} \\
    q^{-n} \left( 1 - q^{-2} \right)
      & \text{if $n \leq r$}.
  \end{cases}
\]

\paragraph{Double sum for Case 2.}
We sum over $(m,n)$ such that
\begin{equation}
  \begin{aligned}
    1 &\leq n \leq \ell + r, \\
    \max\left(n-r, \left\lceil \frac{n-r}{2} \right\rceil+\delta+r \right)
    &\leq m \leq \min(n,\lambda)+\delta+r.
  \end{aligned}
  \label{eq:odd_range2}
\end{equation}
where each $(m,n)$ gives a volume contribution of
\[
  \begin{cases}
    q^{-n - (m-\delta-r)} \left( 1 - q\inv \right)
      & \text{if $m > \delta + r$} \\
    q^{-n} \left( 1 - q^{-2} \right)
      & \text{if $m \le \delta + r$}.
  \end{cases}
\]
Notice that $m \leq \delta + r$ could only occur when $n \leq r$.

\subsubsection{Analysis of Case 3 and 4 assuming $n > 0$ and $v(b) = v(d) = 0$.}
Suppose $\ell \geq 0$ is even.
Using the identity
\[ (au-\bar d \bar u)^2 - \tau^2 = 4a\bar d(1- u\bar u) \]
we agree now to fix the choice of the square root of $\tau$ such that
\begin{equation}
  v(au-\bar d \bar u + \tau) = \lambda - \half \ell \quad\text{and}\quad
  v(au-\bar d \bar u - \tau) = \half \ell.
  \label{eq:tau_choice}
\end{equation}

As before we consider $n > 0$ and $n-r \le m \le n+\delta+r$ as fixed,
and seek to compute the volume of the set of $t$
for which $n = v(1-t\bar t)$ and $\mathbf{1}_{\le r}(\gamma,t,m) = 0$.

From $v(b) = v(d) = 0$ and \eqref{eq:dos}, we have
\[ \ell = 2v(\tau) = 2v(au - \bar d \bar u) < \lambda. \]
This lets us invoke \cite[Lemma 4.7]{ref:AFL} to evaluate $v(4-\Norm(b \pm \tau))$:
we have the calculation
\begin{align*}
  v\left( 4 - \Norm(b+\tau) \right) &= \lambda + \delta - \ell \\
  v\left( 4 - \Norm(b-\tau) \right) &= \delta.
\end{align*}

\paragraph{Double sum for Case 3\ts+.}
Suppose $n > \ell + r$,
$m < n - \frac{\ell}{2} + \delta$, and we choose $\frac{b+\tau}{2a}$.
Then \Cref{lem:quadruple_ineq} gives a nonzero contribution if and only if
\begin{align*}
  \lambda + \delta - \ell = v(4-\Norm(b+\tau)) &\geq n - \frac{\ell}{2} - r \\
  \lambda - \frac{\ell}{2} = v(au-\bar d \bar u + \tau) & \geq m - \delta - r.
\end{align*}
Compiling all seven constraints gives that the valid pairs $(m,n)$ are those for which
\begin{align*}
  \max(1, \ell+r+1) &\leq n \leq -\frac{\ell}{2} + \delta + \lambda + r, \\
  n-r &\leq m \leq \min\left( n+\delta+r, n - \frac{\ell}{2}+\delta - 1,
    \lambda-\frac{\ell}{2}+\delta+r \right)
\end{align*}
which from $\ell, \delta, r \ge 0$ can be simplified to just
\begin{equation}
  \begin{aligned}
    \ell+r+1 &\leq n \leq  -\frac{\ell}{2} + \delta + \lambda + r, \\
    n-r &\leq m \leq \min(n-1, \lambda-r) - \frac{\ell}{2} + \delta
  \end{aligned}
  \label{eq:even_case3_plus}
\end{equation}
Each $(m,n)$ gives a volume contribution of
\[ q^{-n - (n - \frac{\ell}{2} - r)} \left( 1 - q\inv \right). \]

\paragraph{Double sum for Case 3\ts-.}
Suppose $n > \ell + r$,
$m < n - \frac{\ell}{2} + \delta$, and we choose $\frac{b-\tau}{2a}$.
Then \Cref{lem:quadruple_ineq} gives a nonzero contribution if and only if
\begin{align*}
  \delta = v(4-\Norm(b-\tau)) &\geq n - \frac{\ell}{2} - r \\
  \frac{\ell}{2} = v(au-\bar d \bar u - \tau) & \geq m - \delta - r.
\end{align*}
Compiling all seven constraints gives that the valid pairs $(m,n)$ are those for which
\begin{align*}
  \ell + r + 1 &\leq n \leq \frac{\ell}{2}+\delta+r, \\
  n-r &\leq m \leq \min\left( n - \frac{\ell}{2}+\delta - 1,
    \frac{\ell}{2} + \delta + r, n + \delta + r \right)
\end{align*}
This simplifies to
\begin{equation}
  \begin{aligned}
    \ell+r+1 &\leq n \leq \frac{\ell}{2}+\delta+r, \\
    n-r &\leq m \leq \frac{\ell}{2} + \delta + r.
  \end{aligned}
  \label{eq:even_case3_minus}
\end{equation}
As in the previous case, $(m,n)$ gives a volume contribution of
\[ q^{-n - (n - \frac{\ell}{2} - r)} \left( 1 - q\inv \right). \]

\paragraph{Double sum for Case 4\ts+.}
Suppose $n > \ell + r$,
$m \ge n - \frac{\ell}{2} + \delta$
(which implies $m \ge n-r$ since $r \ge 0$ and $0 \le \ell \le 2 \delta$),
and we choose $\frac{b+\tau}{2a}$.
Then \Cref{lem:quadruple_ineq} gives a nonzero contribution if and only if
\begin{align*}
  \lambda &\geq m - \delta - r \\
  \lambda - \frac{\ell}{2} = v(au-\bar d \bar u + \tau) & \geq n - \frac{\ell}{2} - r.
\end{align*}
Rearranging gives that the valid pairs $(m,n)$ are those for which
\begin{equation}
  \begin{aligned}
    \ell + r + 1 &\leq n \leq \lambda + r \\
    n - \frac{\ell}{2} + \delta &\leq m \leq \min(n, \lambda) + \delta + r.
  \end{aligned}
  \label{eq:even_case4_plus}
\end{equation}
Here, each $(m,n)$ gives a volume contribution of
\[ q^{-n - (m - \delta - r)} \left( 1 - q\inv \right). \]

\paragraph{Double sum for Case 4\ts-.}
Suppose $n > \ell + r$,
$m \ge n - \frac{\ell}{2} + \delta$, and we choose $\frac{b-\tau}{2a}$.
Then \Cref{lem:quadruple_ineq} gives a nonzero contribution if and only if
\begin{align*}
  \lambda &\geq m - \delta - r \\
  \frac{\ell}{2} = v(au-\bar d \bar u - \tau) & \geq n - \frac{\ell}{2} - r.
\end{align*}
The latter inequality contradicts the assumption that $n > \ell + r$,
so in fact this case can never occur.

\newpage

\include{orbital2}

\section{Synopsis of the orbital integral
  $\Orb((\gamma, \uu, \vv^\top), \phi \otimes \mathbf{1}_{\OO_F^2 \times (\OO_F^2)^\vee}, s)$
  for $(\gamma, \uu, \vv^\top) \in S_2(F) \times V'_2(F)$ and $\phi \in \HH(S_2(F), K')$}

Throughout this section, $H = \GL_n(F)$ (rather than $H = \GL_{n-1}(F)$)
and $K' = \GL_n(\OO_F)$.
For the concrete calculation, we are mostly interested in the case $n = 2$.

\subsection{Definition}
We will not need to work in the generality of a function
on all of $S_n(F) \times V_n'(F)$, although it could be done.
Instead, it will be enough to define the orbital integral
in the case where our function is of the form
\[ \phi \otimes \mathbf{1}_{\OO_F^n \times (\OO_F^n)^\vee} \]
where $\phi \in \HH(S_n(F), K')$ is the left component, and
the right component is the indicator function defined in the obvious way:
\begin{align*}
  \mathbf{1}_{\OO_F^n \times (\OO_F^n)^\vee} \colon V'_n(F) &\to \{0,1\} \\
  (\uu, \vv^\top) &\mapsto
  \begin{cases}
    1 & \uu \text{ and } \vv^\top \text{ have } \OO_F \text{-entries} \\
    0 & \text{otherwise}.
  \end{cases}
\end{align*}

Then, unsurprisingly from the definition of our action as
\[ h \cdot (\gamma, \uu, \vv^\top) = (h\gamma h\inv, h\uu, \vv^\top h\inv) \]
we analogously define the orbital integral as follows.
\begin{definition}
  For brevity let $\eta(h) \coloneqq \eta(\det h)$ for $h \in H$.
  For $(\gamma, \uu, \vv^\top) \in S_n(F) \times V'_n(F)$,
  $\phi \in \HH(S_n(F), K')$, and $s \in \CC$,
  we define the orbital integral by
  \begin{align*}
    & \Orb((\gamma, \uu, \vv^\top), \phi \otimes \mathbf{1}_{\OO_F^n \times (\OO_F^n)^\vee}, s) \\
    &\coloneqq
    \int_{h \in H} \phi(h\inv \gamma h)
    \mathbf{1}_{\OO_F^n \times (\OO_F^n)^\vee}(h \uu, \vv^\top h^{-1})
    \eta(h) \left\lvert \det(h) \right\rvert_F^{-s} \odif h
  \end{align*}
\end{definition}

\subsection{Basis for the indicator functions in $\HH(S_2(F), K')$}
From now on assume $n = 2$.
This section is almost an exact analog of \Cref{sec:orbital0_hecke_basis},
so we will be slightly terser.
Again set
\[ S_2(F) \coloneqq \left\{ g \in \GL_2(E) \mid g \bar{g} = \id_2 \right\}. \]
We again have a Cartan decomposition indexed by a single integer $r \ge 0$:
\begin{lemma}
  [Cartan decomposition of $S_2(F)$]
  For each integer $r \ge 0$ let
  \[ K'_{S,r} \coloneqq \GL_3(\OO_E) \cdot
    \begin{bmatrix} 0 & \varpi^r \\ \varpi^{-r} & 0 \end{bmatrix} \]
  denote the orbit of
  $\begin{bmatrix} 0 & \varpi^r \\ \varpi^{-r} & 0 \end{bmatrix}$
  under the left action of $\GL_2(\OO_E)$.
  Then we have a decomposition
  \[ S_2(F) = \coprod_{r \geq 0} K'_{S,r}. \]
\end{lemma}
Like last time, $K'_{S,r}$ is the part of $S_2(F)$
for which the most negative valuation among the nine entries is $-r$.
And as before we abbreviate the $r = 0$ term specifically:
\begin{align*}
  K'_S
  &\coloneqq K'_{S,0} \\
  &= \GL_2(\OO_E) \cdot \begin{bmatrix} & 1 \\ 1 \end{bmatrix} \\
  &= \GL_2(\OO_E) \cdot \id_2 = S_2(F) \cap \GL_2(\OO_E).
\end{align*}

Repeating the definition
\[ K'_{S, \le r} \coloneqq S_2(F) \cap \varpi^{-r} \GL_2(\OO_E)
  = K'_{S,0} \sqcup K'_{S,1} \sqcup \dots \sqcup K'_{S,r} \]
we get a basis of indicator functions for the Hecke algebra $\HH(S_2(F), K')$:
\begin{proposition}
  For $r \ge 0$, the indicator functions $\mathbf{1}_{K'_{S, \le r}}$
  form a basis of $\HH(S_2(F), K')$.
\end{proposition}

\subsection{Parametrization of $\gamma$}
From now on assume $n = 2$,
and that $(\gamma, \uu, \vv^\top)$ is regular when viewed
as an element $\begin{bmatrix} \gamma & \uu \\ \vv^\top & 0 \end{bmatrix} \in \GL_3(F)$
(cf.\ \Cref{def:regular}).

\subsubsection{Identifying an orbit representative}
The orbital integral depends only on the $H$-orbit of $(\gamma, \uu, \vv^\top)$.
Consequently, we may assume without loss of generality
(via multiplication by a suitable change-of-basis $h \in H = \GL_2(F)$) that
\[ \uu = \begin{bmatrix} 0 \\ 1 \end{bmatrix}, \qquad
  \vv^\top = \begin{bmatrix} 0 & \theta \end{bmatrix} \qquad \theta \in F. \]
(We know $\uu$ is not the zero vector from the regular condition
applied on $(\gamma, \uu, \vv^\top)$.)

Meanwhile, we will let
$\gamma = \begin{bmatrix} a & b \\ c & d \end{bmatrix} \in \GL_2(F)$
for $a,b,c,d \in F$.
Then, viewed as an element of $\GL_3(F)$ via the embedding we described earlier, we have
\[
  (\gamma, \uu, \vv^\top)
  \mapsto \begin{bmatrix}
    a & b & 0 \\
    c & d & 1 \\
    0 & \theta & 0
  \end{bmatrix} \in \GL_3(F).
\]
Thus, our definition of regular requires that
$\begin{bmatrix} 0 \\ 1 \end{bmatrix}$
is linearly independent from $\begin{bmatrix} b \\ d \end{bmatrix}$
and
$\begin{bmatrix} 0 & \theta \end{bmatrix}$
is linearly independent from $\begin{bmatrix} c & d \end{bmatrix}$.
This is just saying that $b$, $c$, $\theta$ are all nonzero.
We also know that $\gamma \in S_2(F)$, which gives us relations on $a$, $b$, $c$, $d$,
(the same as \cite[equation (7.3.2)]{ref:AFLspherical}); we have
\[
  \begin{bmatrix} 1 & 0 \\ 0 & 1 \end{bmatrix}
  = \begin{bmatrix} a & b \\ c & d \end{bmatrix} \begin{bmatrix} \bar a & \bar b \\ \bar c & \bar d \end{bmatrix}
  \implies
  \begin{aligned}
    \bar b c = b \bar c &= 1 - a \bar a \\
    \text{and } d &= - \bar a c / \bar c = -\bar a b / \bar b.
  \end{aligned}
\]

\begin{remark}
  Note that for each integer $n$, the quantity $\vv^\top \gamma^n \uu$
  is invariant under the action of $h$.
\end{remark}

\subsubsection{Simplification due to the matching of non-quasi-split unitary group}
Like before, we focus on the case where regular $(\gamma, \uu, \vv^\top)$
matches an element in the non-quasi-split unitary group.
\todo{I need to ask Wei exactly what's up here}
This is controlled by the parity of $v(\Delta)$, where
\[ \Delta = \det \left[ \vv^\top \gamma^{i+j} \uu \right]_{0 \le i,j \le n-1}. \]
When $n=2$, for the representatives we described before,
we have
\[ \left[ \vv^\top \gamma^{i+j} \uu \right]_{0 \le i,j \le n-1}
  = \begin{bmatrix} e & de \\ de & bce + d^2e \end{bmatrix} \]
so
\[ \Delta = bce^2 = \frac{b}{\bar b}(1-a \bar a) e^2 . \]
Hence, $v(\Delta)$ is odd if and only if $v(1-a \bar a)$ is odd.
Thus, we restrict attention to the following situation:
\begin{assume}
  We will assume that
  \[ v(1-a \bar a) \equiv 1 \pmod 2. \]
  \label{assume:a_odd}
\end{assume}
In particular, $a$ must be a unit.
And since $d = -\bar a c / \bar c$, it follows $d$ is a unit.
In other words, \Cref{assume:a_odd} gives the direct corollary
\[ v(a) = v(d) = 0. \]

\subsection{Quantities to state the answer in terms of}

\section{Setup of the orbital integral for $S_2(F) \times V_2'(F)$}
\subsection{Iwasawa decomposition}
The overall method is to take the Iwasawa decomposition in $KAN$ form:
every element in $h \in \GL_2(F)$ may be parametrized as
\[ h = k \begin{bmatrix} x_1 & 0 \\ 0 & x_2 \end{bmatrix}
  \begin{bmatrix} 1 & y \\ 0 & 1 \end{bmatrix} \]
where $k \in K' = \GL_2(\OO_F)$, $x_1, x_2 \in \OO_F^\times$ and $y \in \OO_F$.
Because the orbits are invariant under conjugation by $K'$,
the parameter $k$ can be discarded.
The Haar measure integrated over is then given just by
\[ \odif[\times] x_1 \odif[\times] x_2 \odif y \]
i.e.\ we take multiplicative Haar measure on $F^\times$
(normalized so that $\OO_F^\times$ has volume $1$)
and additive Haar measure on $F$
(so $\OO_F$ has volume $1$).

\subsection{Action of upper triangular matrices on $(\gamma, \uu, \vv^\top)$.}
We now compute the action of an arbitrary
\[ h = \begin{bmatrix} x_1 & 0 \\ 0 & x_2 \end{bmatrix}
  \begin{bmatrix} 1 & y \\ 0 & 1 \end{bmatrix} \]
on $(\gamma, \uu, \vv^\top)$.
The main term is given by
\begin{align*}
  h \gamma h^{-1}
  &=
  \begin{bmatrix} x_1 & 0 \\ 0 & x_2 \end{bmatrix}
  \begin{bmatrix} 1 & y \\ 0 & 1 \end{bmatrix}
  \begin{bmatrix} a & b \\ c & d \end{bmatrix}
  \begin{bmatrix} 1 & -y \\ 0 & 1 \end{bmatrix}
  \begin{bmatrix} x_1^{-1} & 0 \\ 0 & x_2^{-1} \end{bmatrix} \\
  &=
  \begin{bmatrix} x_1 & 0 \\ 0 & x_2 \end{bmatrix}
  \begin{bmatrix} cy + a & -cy^2+(d-a)y+b \\ c & -cy+d \end{bmatrix}
  \begin{bmatrix} x_1^{-1} & 0 \\ 0 & x_2^{-1} \end{bmatrix} \\
  &=
  \begin{bmatrix} cy + a & \frac{x_1}{x_2} \cdot \left( -cy^2+(d-a)y+b \right) \\
    \frac{x_2}{x_1} \cdot c & -cy+d \end{bmatrix}
\end{align*}
Meanwhile, we have
\begin{align*}
  h \uu &=
    \begin{bmatrix} x_1 & 0 \\ 0 & x_2 \end{bmatrix}
    \begin{bmatrix} 1 & y \\ 0 & 1 \end{bmatrix}
    \begin{bmatrix} 0 \\ 1 \end{bmatrix}
    = \begin{bmatrix} x_1 y \\ x_2 \end{bmatrix} \\
  \vv^\top h^{-1} &=
    \begin{bmatrix} 0 & e \end{bmatrix}
    \begin{bmatrix} 1 & -y \\ 0 & 1 \end{bmatrix}
    \begin{bmatrix} x_1^{-1} & 0 \\ 0 & x_2^{-1} \end{bmatrix}
    = \begin{bmatrix} 0 & \frac{e}{x_2} \end{bmatrix}.
\end{align*}

\subsection{Description of support}
From now on we fix the notation
\begin{align*}
  n_1 &\coloneqq v(x_1) \\
  n_2 &\coloneqq v(x_2).
\end{align*}

For a given $r \ge 0$, we find that $h$ contributes to the integral exactly
if $h\uu$ and $\vv^\top h\inv$ have $\OO_F$-entries,
and all the entries of $h \gamma h\inv$ are in $\varpi^{-r}\OO_F$.
The former condition is just saying that
\begin{align*}
  v(y) &\ge -n_1 \\
  0 &\le n_2 \le v(e).
\end{align*}

\section{Evaluation of the orbital integral for $S_2(F) \times V'_2(F)$}
\label{sec:orbitalFJ2}


\chapter{The geometric side}
\label{ch:geo}

\section{Rapoport-Zink spaces}
We briefly recall the theory of Rapoport-Zink spaces.
This follows the exposition in \cite[\S4.1]{ref:survey}.

Let $\breve F$ denote the completion of a maximal unramified extension of $F$,
and let $\FF$ denote the residue field of $\OO_{\breve F}$.
Suppose $S$ is a $\Spf \OO_{\breve F}$-scheme.
Then we can consider triples $(X, \iota, \lambda)$ consisting of the following data.
\begin{itemize}
  \ii $X$ is a formal $\varpi$-divisible $n$-dimensional $\OO_F$-module over $S$
  whose relative height is $2n$.

  \ii $\iota \colon \OO_E \to \End(X)$ is an action of $\OO_E$
  such that the induced action of $\OO_F$ on $\Lie X$
  is via the structure morphism $\OO_F \to \SO_S$.

  We require that $\iota$ satisfies the Kottwitz condition of signature $(n-1,1)$,
  meaning that for all $a \in \OO_E$,
  the characteristic polynomial of $\iota(a)$ on $\Lie X$
  is exactly \[ (T-a)^{n-1} (T-\bar a) \in \SO_S[T]. \]

  \ii $\lambda \colon X \to X^\vee$ is a principal $\OO_F$-relative polarization.

  We require that the Rosati involution of $\lambda$
  induces the map $a \mapsto \bar a$ on $\OO_F$
  (i.e.\ the nontrivial automorphism of $\Gal(E/F)$).
\end{itemize}
The triple is called supersingular if $X$ is a supersingular strict $\OO_F$-module.

For each $n \ge 1$, over $\FF$
we choose a supersingular triple $(\XX_n, \iota_{\XX_n}, \lambda_{\XX_n})$;
it's unique up to $\OO_E$-linear quasi-isogeny compatible with the polarization,
and refer to it as the \emph{framing object}.
We also let \[ \EE \coloneqq \XX_1. \]

We can now define the Rapoport-Zink space:
\begin{definition}
  For each $n \ge 1$, we let $\RZ_n$ denote the
  functor over $\Spf \OO_{\breve F}$ defined as follows.
  Let $S$ be an $\Spf \OO_{\breve F}$ scheme, and let
  $\ol S \coloneqq S \times_{\Spf \OO_{\breve F}} \Spec \FF$
  For every $\Spf \OO_{\breve F}$ scheme, we let $\RZ_n(S)$
  be the set of isomorphism classes of quadruples
  \[ (X, \iota, \lambda, \rho) \]
  where $(X, \iota, \lambda)$ is one of the triples as we described, and
  \[ \rho \colon X \times_S \ol S \to \XX_n \times_{\Spec \FF} \ol S \]
  is a \emph{framing}, meaning it is an height zero $\OO_F$-linear quasi-isogeny
  and satisfies
  \[ \rho^\ast((\lambda_{\XX_n})_{\ol S}) = \lambda_{\ol S}. \]
\end{definition}
Then $\RZ_n$ is formally smooth over $\OO_{\breve F}$ of relative dimension $n-1$.

Henceforth, we also make the following abbreviation.
\begin{definition}
  For integers $m$ and $n$
  \[ \RZ_{m,n} \coloneq \RZ_{m} \times_{\Spf \OO_{\breve F}} \RZ_n. \]
\end{definition}

\section{A realization of the non-split Hermitian space $\VV_n$ of dimension $n$}
At the same time, we can define the following Hermitian space.
\begin{definition}
  For each $n \ge 1$, let
  \[ \VV_n \coloneqq \Hom_{\OO_E}^\circ (\EE, \XX_n) \]
  which we call the space of special homomorphisms.
  When endowed with the form
  \[ \left< x,y \right> = \lambda_{\EE}^{-1} \circ y^\vee \circ \lambda_{\XX_n} \circ x
    \in \End_{E}^\circ(\EE) \simeq E \]
  it becomes an $n$-dimensional $E/F$-Hermitian space.
\end{definition}
\begin{proposition}
  Up to isomorphism, $\VV_n$ is the unique $n$-dimensional
  nondegenerate non-split $E/F$-Hermitian space.
\end{proposition}
\begin{proof}
  \todo{ref}
\end{proof}

\section{Intersection numbers for the group version of AFL for the full spherical Hecke algebra}
Here we reproduce the definition of the intersection number used in \Cref{conj:inhomog}.

Compared to the formulation of the group version and semi-Lie version of the AFL,
the intersection number requires the introduction of a
\emph{Hecke operator} $\TT_{\varphi}$ for an element
\[ \varphi \in \HH(G^\flat \times G, K^\flat \times K) \]
as introduced in \cite{ref:AFLspherical}.
This definition is too involved to reproduce here in its entirety,
we give a summary for this special cases in which we need.

First consider the given $f \in \HH(G, K)$.
The main work of the construction is to define another
formal scheme $\mathcal{T}_{\mathbf{1}_{K^\flat} \otimes f}$ (see \cite[\S6.1]{ref:AFLspherical})
together with two projection maps
\begin{center}
\begin{tikzcd}
  & \ar[ld] \mathcal T_{\mathbf{1}_{K^\flat} \otimes f} \ar[rd] & \\
  \RZ_{n-1,n} && \RZ_{n-1,n}
\end{tikzcd}
\end{center}
This definition is carried out in \cite[\S5]{ref:AFLspherical},
by defining it first for so-called \emph{atomic elements} of the spherical Hecke algebra,
which form basis elements of a certain presentation of this Hecke algebra
as the unitary group for a polynomial algebra;
we refer the reader to \emph{loc. cit.}~for the full details.

Now, take the natural closed embedding
\[ \RZ_{n-1} \to \RZ_n \]
and let
\[ \Delta \colon \RZ_{n-1} \hookrightarrow \RZ_{n-1,n} \]
be the associated graph morphism.
Then we denote by $\iota \colon \Delta(\RZ_n) \hookrightarrow \RZ_{n-1,n}$
the inclusion mapping of $\Delta$.
Once this is done, consider then the diagram
\begin{center}
\begin{tikzcd}
  & \pi_1^\ast(\Delta_{\RZ_{n-1,n}}) \ar[ld] \ar[r] \ar[rd]
    & \ar["\pi_1" near end, ld] \mathcal T_{\mathbf{1}_{K^\flat} \otimes f}
      \ar["\pi_2" near end, rd] \\
  \Delta_{\RZ_{n-1,n}} \ar[r, "\iota"', hook] & \RZ_{n-1,n}
  & (\pi_2)_\ast(\pi_1^\ast(\Delta_{\RZ_{n-1,n}})) \ar[r] & \RZ_{n-1,n}.
\end{tikzcd}
\end{center}
That is, one takes the pullback of
$\Delta_{\RZ_{n-1,n}} \xhookrightarrow{\iota} \RZ_{n-1,n}$
along the projection
\[ \RZ_{n-1,n} \xleftarrow{\pi_1} \mathcal{T}_{\mathbf{1}_{K^\flat} \otimes f} \]
and then takes the pushforward along the other projection
\[ \mathcal{T}_{\mathbf{1}_{K^\flat} \otimes f} \xrightarrow{\pi_2} \RZ_{n-1,n}. \]
\begin{definition}
  Set
  \[
    \TT_{\mathbf{1}_{K^\flat} \otimes f} (\Delta_{\RZ_{n-1,n}})
    \coloneqq (\pi_2)_\ast(\pi_1^\ast(\Delta_{\RZ_{n-1,n}})).
  \]
\end{definition}
This is the part of the intersection number depending on $f$
(or rather $\TT_{\mathbf{1}_{K^\flat} \otimes f}$).
As for our $g \in G\rs$, we simply consider the translation $(1,g) \cdot \Delta_{\RZ_{n-1,n}}$.
The intersection number is then defined as by taking the intersection
of these two objects using the derived tensor product $\jiao$ of the structure sheaves.
\begin{definition}
  [{\cite[(6.1.1)]{ref:AFLspherical}}]
  We define the intersection number in \Cref{conj:inhomog}
  \[
    \Int((1,g), \mathbf{1}_{K^\flat} \otimes f)
    \coloneqq \chi_{\RZ_{n-1,n}} \left(
      \SO_{\TT_{\mathbf{1}_{K^\flat} \otimes f} (\Delta_{\RZ_{n-1}})}
      \jiao_{\SO_{\RZ_{n-1,n}}} \SO_{(1,g) \cdot \Delta_{\RZ_{n-1}}} \right).
  \]
\end{definition}
Here $\chi$ denotes the Euler-Poincar\'{e} characteristic,
meaning that if $X$ is a formal scheme over $\Spf \OO_{\breve F}$
then given a finite complex $\mathcal{F}$ of $\SO_X$-modules we set
\[ \chi_X(\mathcal{F}) = \sum_i \sum_j (-1)^{i+j}
  \operatorname*{len}_{\OO_{\breve F}} H^j(X, H_i(\mathcal F)) \]
provided all the lengths are finite.

\section{Intersection numbers for the semi-Lie version of AFL for the full spherical Hecke algebra}
Now we continue to define an intersection number needed for the proposed
\Cref{conj:semi_lie_spherical} from earlier.
The definition mirrors the one given in the last section.
Here we reproduce the definition of the intersection number used in \Cref{conj:inhomog}.

We work here with $\RZ_{n,n}$ rather than $\RZ_{n-1,n}$.
The change is that we need to incorporate the new $u \in \VV_n$ that was not present before.
In order to do this one considers a certain relative Cartier divisor $\mathcal Z(u)$
on $\RZ_n$ for each nonzero $u \in \VV_n$.
This divisor was defined by Kudala and Rapport in \cite{ref:KR}
and accordingly we call it a \emph{KR-divisor} following \cite[\S4.3]{ref:survey}.
The definition is given as follows.
\begin{definition}
  Since $\RZ_1 \cong \Spf \OO_{\breve F}$, the formal $\OO_F$-module $\EE = \XX_1$
  has a unique lifting called its \emph{canonical lifting}, which we
  denote by the triple $(\mathcal{E}, \iota_{\mathcal{E}}, \lambda_{\mathcal E})$.
  Then the KR-divisor $\mathcal Z(u)$ is the locus where the quasi-homomorphism
  $\EE \to \XX_n$ lifts to a homomorphism from $\mathcal{E}$ to the universal object over $\RZ_n$.
\end{definition}
Colloquially, the KR-divisor $\mathcal Z(u)$ can be thought of as the set of diagrams of the form
\begin{center}
\begin{tikzcd}
  \ol{\mathcal{E}}\ar[r] \ar[d, dash] & \mathcal{X}_n \ar[d, dash] \\
  \EE \ar[r, "u"] & \XX_n.
\end{tikzcd}
\end{center}
Note also by the definition that $g \mathcal Z(u) = \mathcal Z (gu)$.

The main change is then that we can consider $\Delta_{\mathcal Z(u)}$ as the image of
\[ \mathcal Z(u) \hookrightarrow \RZ_n \xrightarrow{\Delta} \RZ_{n,n} \]
where $\Delta \colon \RZ_n \to \RZ_{n,n}$ now denotes the diagonal map.
If one defines an appropriate space $\mathcal T_{f}$ for $f \in \HH(G, K)$ together with
\begin{center}
\begin{tikzcd}
  & \ar[ld] \mathcal T_{f} \ar[rd] & \\
  \RZ_{n,n} && \RZ_{n,n}
\end{tikzcd}
\end{center}
then one can then repeat the diagram from before:
\begin{center}
\begin{tikzcd}
  & \pi_1^\ast(\Delta_{\mathcal Z(u)}) \ar[ld] \ar[r] \ar[rd]
    & \ar["\pi_1" near end, ld] \mathcal T_{f}
      \ar["\pi_2" near end, rd] \\
  \Delta_{\mathcal Z(u)} \ar[r, "\iota"', hook] & \RZ_{n,n}
  & (\pi_2)_\ast(\pi_1^\ast(\Delta_{\mathcal Z(u)})) \ar[r] & \RZ_{n,n}.
\end{tikzcd}
\end{center}
In other words, we again take a pullback followed by a pushforward
but this time of $\Delta_{\mathcal Z(u)} \hookrightarrow \RZ_{n,n}$.
This lets us write an analogous definition:
\begin{definition}
  Set
  \[
    \TT_f (\Delta_{\mathcal Z(u)})
    \coloneqq (\pi_2)_\ast(\pi_1^\ast(\Delta_{\mathcal Z(u)})).
  \]
\end{definition}
Meanwhile to replace $(1,g) \Delta_{\RZ_{n,n-1}}$, we let
\[ \Gamma_g \subseteq \RZ_{n,n} \]
denote the graph of the automorphism of $\RZ_n$ induced by $g$.
This finally allows us to write a definition of the intersection number in the semi-Lie case:
\begin{definition}
  We define the intersection number in \Cref{conj:semi_lie_spherical} as
  \[
    \Int((g,u), f)
    \coloneqq \chi_{\RZ_{n,n}} \left(
      \SO_{\TT_f(\Delta_{\ZD(u)})}
      \jiao_{\SO_{\RZ_{n,n}}} \SO_{\Gamma_g} \right).
  \]
\end{definition}

\section{An analogy between the geometric and analytic sides}
With the intersection number now defined for \Cref{conj:semi_lie_spherical},
we provide some philosophical discussion about their connection.
All of this is for philosophical motivation only,
and is not meant to formally assert any definitions or results.

Let us first discuss the case of \Cref{conj:semi_lie_spherical} when
$f = \mathbf{1}_K$


\printbibliography[title=References,heading=bibintoc]

\appendix

\end{document}
