\section{Introduction}
Throughout this whole paper, $p > 2$ is a prime,
$F$ is a finite extension of $\QQ_p$,
and $E/F$ is an unramified quadratic field extension.

\subsection{History and motivation}
We briefly review the history and context

\subsubsection{The GGP conjectures}
In modern arithmetic geometry, a common theme is that there are deep connections
between geometric data with the values of related $L$-functions.
So our story begins with two results published at similar times:
\begin{itemize}
  \ii In 1986, Gross-Zagier \cite{ref:gross_zagier} proved a formula
  relating the height of Heegner points
  on certain modular curves to the derivative at $s=1$ of certain $L$-functions.

  \ii In 1985, Waldspurger \cite{ref:waldspurger} showed a formula
  relating the nonvanishing of an automorphic period integral
  to the central value of the same $L$-functions.
  Later, a conjecture that generalizes Waldspurger's formula
  was proposed by Gross-Prasad in \cite{ref:GP1,ref:GP2}.
\end{itemize}
These two results then lead naturally to a series of conjectures
now known as the Gan-Gross-Prasad (GGP) conjectures,
which were proposed in 2012 in \cite{ref:GGP}.
The GGP conjectures generalize both the results above:
they generalize the Gross-Zagier formula by replacing the modular curve
with higher-dimensional Shimura varieties,
while generalizing the Gross-Prasad conjecture to different classical groups.
Specifically, the GGP conjectures predict the nonvanishing of a period integral
based on the values of the $L$-function of a certain cuspidal automorphic representation.

In 2011, Jacquet-Rallis \cite{ref:JR} proposed an approach to the Gross-Prasad conjectures
for unitary groups via a relative trace formula (RTF).
The idea is to compare a RTF for the general linear group to one for a unitary group.
This approach relies on a so-called \emph{fundamental lemma},
which links values of certain orbital integrals
over two reductive groups over a non-Archimedean local field.
\todo{make something precise here}
The fundamental lemma has since been proved completely;
a local proof was given by Beuzart-Plessis \cite{ref:BeuzartPlessis}
while a global proof was given for large characteristic by W.\ Zhang \cite{ref:Wei2021}.

\subsubsection{The arithmetic GGP conjectures.}
An arithmetic analogue of the Gan-Gross-Prasad conjectures,
which we henceforth refer to as \emph{arithmetic GGP}, can also be formulated.
Rather than period integrals,
one instead considers intersection numbers of cycles on some Shimura varieties.
Specifically, if one considers the Shimura variety associated to a classical group,
the arithmetic GGP conjecture predicts a relation between intersection numbers
on this Simura variety with the central derivative of automorphic $L$-functions.

By analogy to the work Jacquet-Rallis \cite{ref:JR},
the arithmetic GGP conjectures should have a corresponding
\emph{arithmetic fundamental lemma} (henceforth AFL),
which was proposed by W.\ Zhang \cite{ref:AFL}.
The arithmetic fundamental lemma then relates the derivative of a certain orbital integral
to arithmetic intersection numbers on a Rapoport-Zink formal moduli space.
The AFL in \cite{ref:AFL} has since been proven over $p$-adic fields for any prime $p$ in
Mihatsch-Zhang \cite{ref:MZ2021}, W.\ Zhang \cite{ref:Wei2021}, Z.\ Zhang \cite{ref:Zhiyu}.


\subsection{Roadmap}

\subsection{Acknowledgmetns}
